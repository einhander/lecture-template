%Some cool feature here, feel free to add some. 
% =======================================================================
% ========= Большой превью в заметках ===================================
% =======================================================================
\setbeamertemplate{note page}%
{%
	{%
			\scriptsize
			\insertvrule{.5\paperheight}{white}
			\vskip-.5\paperheight%
			\nointerlineskip%
			\vbox{
				\hfill\insertslideintonotes{0.5}
				\hskip-1cm\hskip0pt%
				\vskip-0.5\paperheight%
				\nointerlineskip%
				\begin{pgfpicture}{0cm}{0cm}{0cm}{0cm}
					\begin{pgflowlevelscope}{\pgftransformrotate{90}}
						{\pgftransformshift{\pgfpoint{-2cm}{0.2cm}}%
							\pgftext[base,left]{\footnotesize\the\year-\ifnum\month<10\relax0\fi\the\month-\ifnum\day<10\relax0\fi\the\day}}
					\end{pgflowlevelscope}
				\end{pgfpicture}
			}
			\nointerlineskip%
			\vbox to 0.5\paperheight{\vskip0.5em%
				\hbox{\insertshorttitle[width=3cm]}%
				\begin{minipage}{3cm}
					\insertsection \par%
					\insertsubsection%
				\end{minipage}
				\vskip0.5em
				\textbf{Заголовок:} \par
				\hbox{\insertshortframetitle[width=3cm]}%
				\par
				\textbf{Слайд:} \par
				\insertframenumber.\arabic{slidenumber}{}/ \inserttotalframenumber \par
				\vfil}%
		}%
	\vskip-0.025\paperheight%
	\nointerlineskip%
	\insertnote%
}
%================Белый цвет фона в минислайде=================================
\makeatletter
\renewcommand{\insertslideintonotes}[1]{{%
			\begin{pgfpicture}{0cm}{0cm}{#1\paperwidth}{#1\paperheight}
				\begin{pgflowlevelscope}{\pgftransformscale{#1}}%
					\color{white} % меняем тут
					\pgfpathrectangle{\pgfpointorigin}{\pgfpoint{\paperwidth}{\paperheight}}
					\pgfusepath{fill}
					\color{black} %цвет шрифта
					{\pgftransformshift{\pgfpoint{\beamer@origlmargin}{\footheight}}\pgftext[left,bottom]{\copy\beamer@frameboxcopy}}
				\end{pgflowlevelscope}
			\end{pgfpicture}%
		}}
\makeatother
%=====================Конец==================================================

%============================Подсчет количества слайдов с оверлееми===========
\newcounter{slidenumber}

\defbeamertemplate*{footline}{infolines theme frame plus slide}{
	\setcounter{slidenumber}{\insertpagenumber}%
	\addtocounter{slidenumber}{-\insertframestartpage}%
	\addtocounter{slidenumber}{1}%
	\leavevmode%
	\hbox{%
		\begin{beamercolorbox}[wd=.333333\paperwidth,ht=2.25ex,dp=1ex,center]{author in head/foot}%
			\usebeamerfont{author in head/foot}\insertshortauthor~~(\insertshortinstitute)
		\end{beamercolorbox}%
		\begin{beamercolorbox}[wd=.333333\paperwidth,ht=2.25ex,dp=1ex,center]{title in head/foot}%
			\usebeamerfont{title in head/foot}\insertshorttitle
		\end{beamercolorbox}%
		\begin{beamercolorbox}[wd=.333333\paperwidth,ht=2.25ex,dp=1ex,right]{date in head/foot}%
			\usebeamerfont{date in head/foot}\insertshortdate{}\hspace*{2em}
			\insertframenumber.\arabic{slidenumber}{}/ \inserttotalframenumber\hspace*{2ex}
		\end{beamercolorbox}}%
	\vskip0pt%
}
%require to set below
%\setbeamertemplate{footline}[infolines theme frame plus slide]



%====================Буллет в note[item]======================================
\makeatletter
\def\beamer@setupnote{%
	\gdef\beamer@notesactions{%
		\beamer@outsideframenote{%
			\beamer@atbeginnote%
			\beamer@notes%
			\ifx\beamer@noteitems\@empty\else
				\begin{itemize}\itemsep=0pt\parskip=0pt%
					\beamer@noteitems%
				\end{itemize}%
			\fi%
			\beamer@atendnote%
		}%
		\gdef\beamer@notesactions{}%
	}
}

\makeatother
%=================================Конец=====================================



%===========================   Выравнивание по ширине в block  ===========
\addtobeamertemplate{block begin}{}{\justifying}
\renewcommand{\raggedright}{\leftskip=0pt \rightskip=0pt plus 0cm} %global

%============================== Библиография занимает меньше места ======
\setbeamertemplate{bibliography entry title}{}
\setbeamertemplate{bibliography entry location}{}
\setbeamertemplate{bibliography entry note}{}
\setbeamertemplate{bibliography item}{\insertbiblabel}
%===========================================================================

%===========Заменяем библиографию с квадратных скобок на точку
\makeatletter
\renewcommand{\@biblabel}[1]{#1.}
\makeatother
%============================================================

%============ Черный шрифт для библиографии  ==================================
\setbeamercolor{bibliography item}{fg=black}
\setbeamercolor*{bibliography entry title}{fg=black}
\setbeamercolor*{bibliography entry author}{fg=black}
\setbeamercolor*{bibliography entry journal}{fg=black}
\setbeamercolor*{bibliography entry location}{fg=black}
%==============================================================================


%================== Общие макросы и команды ==========================

% просто чтобы не писать много буковов
\newcommand{\tb}[1]{\textsubscript{#1}}
\newcommand{\tp}[1]{\textsuperscript{#1}}

%============ The fullFrameMovie command defenition ==========================
% from early pdfpc sty 
% Package: textpos is required for textblock*
\usepackage[absolute,overlay]{textpos}


% fullFrameMovie
%
% Arguments:
%
%   [optional]: movie-options, seperated by &
%       Supported options: loop, start=N, end=N, autostart
%   Default: autostart&loop
%
%   1. Movie file
%   2. Poster image
%   3. Any text on the slide, or nothing (e.g. {})
%
% Example:
%   \fullFrameMovie[loop&autostart]{apollo17.avi}{apollo17.jpg}{\copyrightText{Apollo 17, NASA}}
%
\newcommand{\fullFrameMovie}[4][autostart&loop]
{
	{
			\setbeamercolor{background canvas}{bg=black}


			% to make this work for both horizontally filled and vertically filled images, we create an absolutely
			% positioned textblock* that we force to be the width of the slide.
			% we then place it at (0,0), and then create a box inside of it to ensure that it's always 95% of the vertical
			% height of the frame.  Once we have created an absolutely positioned and sized box, it doesn't matter what
			% goes inside -- it will always be vertically and horizontally centered
			\frame[plain]
			{
				\begin{textblock*}{\paperwidth}(0\paperwidth,0\paperheight)
					\centering
					\vbox to 0.99\paperheight {
						\vfil{
							\href{run:#2?autostart&#1}{\includegraphics[width=\paperwidth,height=0.95\paperheight,keepaspectratio]{#3}}
						}
						\vfil
					}
				\end{textblock*}
				#4
			}
		}
}

\newcommand{\fullFrameMultimedia}[3]
{
	{
			\setbeamercolor{background canvas}{bg=black}
			\usebackgroundtemplate[default]


			% to make this work for both horizontally filled and vertically filled images, we create an absolutely
			% positioned textblock* that we force to be the width of the slide.
			% we then place it at (0,0), and then create a box inside of it to ensure that it's always 95% of the vertical
			% height of the frame.  Once we have created an absolutely positioned and sized box, it doesn't matter what
			% goes inside -- it will always be vertically and horizontally centered
			\frame[plain]
			{
				\begin{textblock*}{\paperwidth}(0\paperwidth,0\paperheight)
					\centering
					\vbox to 0.99\paperheight {
						\vfil{
							% \href{run:#2?autostart&#1}{\includegraphics[width=\paperwidth,height=0.95\paperheight,keepaspectratio]{#3}}
							\movie[width=\paperwidth,height=0.95\paperheight]{\includegraphics[width=\paperwidth,keepaspectratio]{#2}}{#1}

						}
						\vfil
					}
				\end{textblock*}
				#3
			}
		}
}
% inlineMovie
%
% Arguments:
%
%   [optional]: movie-options, seperated by &
%       Supported options: loop, start=N, end=N, autostart
%   Default: autostart&loop
%
%   1. Movie file
%   2. Poster image
%   3. size command, such as width=\textwidth
%
% Example:
%   \inlineMovie[loop&autostart&start=5&stop=12]{apollo17.avi}{apollo17.jpg}{height=0.7\textheight}
%
\newcommand{\inlineMovie}[4][autostart&loop]
{
	\href{run:#2?#1}{\includegraphics[#4]{#3}}
}


% copyrightText
%
% Produces small text on the right side of the screen, useful for
% stating copyright or other small notes in movies or images
%
% Arguments:
%
%   [optional]: text color
%       Default: white
%
%   1. Text to be displayed
%
% Example:
%   \copyrightText{Full frame image of: Apollo 17, NASA}
%
\newcommand\copyrightText[2][white]{%
	\begin{textblock*}{\paperwidth}(0\paperwidth,.97\paperheight)%
		\hfill\textcolor{#1}{\tiny#2}\hspace{20pt}
	\end{textblock*}
}

% fullFrameImageZoomed
%
% Produces a slide that contains a full frame image.  Scales down the image
% to fit if the aspect ratio of the slide does not match the image.
%
% Arguments:
%
%   [optional]: color of text on page
%       Default: white
%
%   1. Path to image file
%   2. Any additional content on the frame
%
% Example:
%   \fullFrameImageZoomed{apollo17.jpg}{\copyrightText{Full frame image of: Apollo 17, NASA}}
%
\newcommand{\fullFrameImage}[3][white]
{
	{
			\setbeamercolor{normal text}{bg=black,fg=#1}


			% to make this work for both horizontally filled and vertically filled images, we create an absolutely
			% positioned textblock* that we force to be the width of the slide.
			% we then place it at (0,0), and then create a box inside of it to ensure that it's always 95% of the vertical
			% height of the frame.  Once we have created an absolutely positioned and sized box, it doesn't matter what
			% goes inside -- it will always be vertically and horizontally centered
			\frame
			{
				\begin{textblock*}{\paperwidth}(0\paperwidth,0\paperheight)
					\centering
					\vbox to 0.95\paperheight {
						\vfil{
							\includegraphics[width=\paperwidth,height=0.95\paperheight,keepaspectratio]{#2}
						}
						\vfil
					}
				\end{textblock*}
				#3
			}
		}
}

% fullFrameImageZoomed
%
% Produces a slide that contains a full frame image.  If the aspect ratio
% of the image does not match the slide, it crops the image.
%
% Arguments:
%
%   [optional]: color of text on page
%       Default: black
%
%   1. Path to image file
%   2. Any additional content on the frame
%
% Example:
%   \fullFrameImageZoomed{apollo17.jpg}{\copyrightText{Full frame image of: Apollo 17, NASA}}
%
\newcommand{\fullFrameImageZoomed}[3][black]
{
	{
			\usebackgroundtemplate{\includegraphics[height=\paperheight]{#2}}
			\setbeamercolor{normal text}{bg=black,fg=#1}
			\frame
			{
				#3
			}
		}
}

% Уменьшает вертикальное расстояние в equation
\makeatletter
\g@addto@macro\normalsize{%
	\setlength\belowdisplayskip{-0pt}
	\setlength\abovedisplayskip{-1pt}
}
\makeatother


% handdrawning decorations
% calc, decorations.pathmorphing, patterns,
%TODO write help
\pgfdeclaredecoration{penciline}{initial}{
	\state{initial}[width=+\pgfdecoratedinputsegmentremainingdistance,
		auto corner on length=1mm,]{
		\pgfpathcurveto%
		{% From
			\pgfqpoint{\pgfdecoratedinputsegmentremainingdistance}
			{\pgfdecorationsegmentamplitude}
		}
		{%  Control 1
			\pgfmathrand
			\pgfpointadd{\pgfqpoint{\pgfdecoratedinputsegmentremainingdistance}{0pt}}
			{\pgfqpoint{-\pgfdecorationsegmentaspect
					\pgfdecoratedinputsegmentremainingdistance}%
				{\pgfmathresult\pgfdecorationsegmentamplitude}
			}
		}
		{%TO 
			\pgfpointadd{\pgfpointdecoratedinputsegmentlast}{\pgfpoint{1pt}{1pt}}
		}
	}
	\state{final}{}
}


% Уменьшает отступы в листинге
\makeatletter
\def\nbhline{%
	\noalign{\ifnum0=`}\fi
		\penalty\@M%
		\futurelet\@let@token\LT@@nobreakhline}
	\def\LT@@nobreakhline{%
		\ifx\@let@token\hline
			\global\let\@gtempa\@gobble%
			\gdef\LT@sep{\penalty\@M\vskip\doublerulesep}% <-- change here
		\else
			\global\let\@gtempa\@empty%
			\gdef\LT@sep{\penalty\@M\vskip-\arrayrulewidth}% <-- change here
		\fi
		\ifnum0=`{\fi}%
	\multispan\LT@cols%
	\unskip\leaders\hrule\@height\arrayrulewidth\hfill\cr
	\noalign{\LT@sep}%
	\multispan\LT@cols%
	\unskip\leaders\hrule\@height\arrayrulewidth\hfill\cr
	\noalign{\penalty\@M}%
	\@gtempa}
\makeatother


