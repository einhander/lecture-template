%Here some packets for beamer presentations. Put needed packages here
%\documentclass[pdf,hyperref={unicode},xcolor={x11names},svgnames]{beamer} % 10.2022 vimtex fail to detect master tex file, so it moved to main file
\usepackage[T2A]{fontenc}
\usepackage[utf8]{inputenc}
\usepackage[english,russian]{babel}
\usepackage{amssymb,
	amsfonts,
	array,
	wasysym,
	%fixltx2e,
	amsmath,
	empheq,         % group equation
	mathtext,
	gensymb,
	textcomp,        % obsolate
	% cite,            % removed for biblatex
	multirow,
	enumerate,
	float,
	icomma,
	xcolor,
	fancybox,
	soulutf8,
	csquotes,
	minted,
	multimedia, %\movie[width=200pt, height=150pt]{example video}{MRauma.mpg}
	booktabs,
	indentfirst,
	rotating,
	fontawesome, % занчки ютуба и стима
	tabularx,
	tabulary,
	pdfpc,      % presentation helper
	pgf
}

%\usefonttheme[onlymath]{serif} %cm math font  
\usefonttheme{professionalfonts}

\usepackage[per-mode=symbol]{siunitx} % Provides the \SI{}{} command for typesetting SI units, набор значений единиц измерения

% Позволяет использовать конструкцию где
\usepackage{eqexpl}

\usepackage[most]{tcolorbox}
\usepackage{colortbl}
%добавить в includegraphics[width=...frame]{image}

\usepackage{graphicx} %хотим вставлять в рисунки?
\graphicspath{{images/}{inc/}}%путь к рисункам

\usepackage{enumitem}
\setitemize{label=\usebeamerfont*{itemize item}%
	\usebeamercolor[fg]{itemize item}
	\usebeamertemplate{itemize item}, topsep=0pt}
\setlist[enumerate]{label*=\arabic*. , leftmargin=15pt }


\usepackage[
	parentracker=true,
	backend=biber,
	hyperref=auto,
	language=auto,
	autolang=other, % иногда даетм много места в скобках
	citestyle=gost-numeric,
	defernumbers=true,
	bibstyle=gost-numeric,
]
{biblatex}

\DeclareSourcemap{
	\maps[datatype=bibtex]{
		\map{
			\step[fieldset=language,fieldvalue=english] % fill empty languge field
			\step[fieldsource=language,fieldtarget=langid, final] % fill langid with language value
			%\step[fieldset=langid, fieldvalue={,}, append]
			%\step[fieldset=keywords, origfieldval, append]
		}
	}
}
\addbibresource{biblio/jabref.bib}
\addbibresource{biblio/activ.bib}
\addbibresource{biblio/my.bib}
\addbibresource{biblio/ochistka.bib}
%\printbibliography

% \setbeamercolor{bibliography entry title}{fg=blue!50!cyan}
% \setbeamercolor{bibliography entry author}{fg=violet}
% \setbeamercolor{bibliography entry location}{fg=green}
\setbeamercolor{bibliography entry note}{fg=black}

%============Химия=========================================================================
\usepackage[version=4]{mhchem}
\usepackage{chemfig} % рисование структурных формул в химии <<настоящий ад и вынос мозга>>
\makeatletter %использование mhchem для строчных атомов в структурных формулах
\def\CF@node@content{%
	\expandafter\expandafter\expandafter
	\printatom\expandafter\expandafter\expandafter
	{\csname atom@\number\CF@cnt@atomnumber\endcsname}%
	\ensuremath{\CF@node@strut}%
}
\makeatother
\setchemfig{
	double bond sep=0.35700 em,
	atom sep=1.78500 em,
	bond offset=0.18265 em,
	bond style={line width =0.06642 em}
}
%============================Позволяет рисовать полимеры===========
\newcommand\setpolymerdelim[2]{\def\delimleft{#1}\def\delimright{#2}}
\def\makebraces[#1,#2]#3#4#5{%
	\edef\delimhalfdim{\the\dimexpr(#1+#2)/2}%
	\edef\delimvshift{\the\dimexpr(#1-#2)/2}%
	\chemmove{%
		\node[at=(#4),yshift=(\delimvshift)]
		{$\left\delimleft\vrule height\delimhalfdim depth\delimhalfdim
				width0pt\right.$};%
		\node[at=(#5),yshift=(\delimvshift)]
		{$\left.\vrule height\delimhalfdim depth\delimhalfdim
				width0pt\right\delimright_{\rlap{$\scriptstyle#3$}}$};}}
%=============Пример==================================
%%\setpolymerdelim()% выбор типа скобок
%%Polyéthylène:
%%\chemfig{\vphantom{CH_2}%используется для поднятия связи на нормальное расстояние от базовой линии
%%	-[@{op,.75}]CH_2-CH_2-[@{cl,0.25}]}
%%\makebraces[5pt,5pt]{\!\!n=13}{op}{cl}
%===============конец химии=======================


%=======Заметки в презентации====
\usepackage{pgfpages}
%\setbeameroption{show notes on second screen=bottom} % расположение заметок, закоментить для отключения 
\setbeamerfont{note page}{size=\footnotesize} %шрифт
%\note{My text} % example
%\note<2> [item]{My enumerate} % example
%===================
%=================  Минд мапы  ====================================
\usepackage{tikz}
\usetikzlibrary{
	mindmap,
	positioning,
	arrows,
	shapes,
	shapes.geometric,
	shapes.callouts,
	shapes.arrows,
	calc,
	decorations.pathmorphing,
	patterns,
	shadows}
%============  Набор физических величин  ==================================
\usepackage{siunitx} % Provides the \SI{}{} command for typesetting SI units, набор значений единиц измерения
\sisetup{range-phrase=--,range-units = single,locale = DE} % no-russian :((

%============  Выравнивание в тексте   ======================
\usepackage{ragged2e}
\justifying

\usepackage[framemethod=TikZ]{mdframed} % красивые боксы вокруг слов


\usepackage[export]{adjustbox}% рамка вокруг рисунков

\makeatletter
%\patchcmd{\@listI}{\itemsep3\p@}{\itemsep0em}{}{}
\renewcommand{\@listI}{%
	\leftmargin=10pt
	\rightmargin=0pt
	\labelsep=5pt
	\labelwidth=20pt
	\itemindent=0pt
	\listparindent=0pt
	\topsep=8pt plus 2pt minus 4pt
	\partopsep=2pt plus 1pt minus 1pt
	\parsep=0pt plus 1pt
	\itemsep=\parsep}
\makeatother

\usepackage{listings}                % пакет для набора исходных текстов программ
\usepackage{lstautogobble}           % доп пакет для игнорирования начальных пробелов
% \usepackage{listingsutf8}           % пакет для набора исходных текстов программ
\usepackage{xcolor}                  % пакет для цвета, включен здесь если ранее не включен

\usepackage[outline]{contour}
\contourlength{1.2pt}

%=================  Конец пакетов  =============================================
% Добавляет содержание перед секцией
\AtBeginSection[]
{\begin{frame}
		\frametitle{Содержание}
		\tableofcontents[currentsection]
	\end{frame} }






%====================  Colors  ===========================================================
\definecolor{airforceblue}{rgb}{0.36, 0.54, 0.66}       %blue
\definecolor{aliceblue}{rgb}{0.94, 0.97, 1.0}           %white
\definecolor{darkslateblue}{rgb}{0.28, 0.24, 0.55}      %blue
\definecolor{antiquefuchsia}{rgb}{0.57, 0.36, 0.51}     %purple
\definecolor{darkraspberry}{rgb}{0.53, 0.15, 0.34}      %purple

\definecolor{debianred}{rgb}{0.84, 0.04, 0.33}          %red
\definecolor{darkred}{rgb}{0.55, 0.0, 0.0}              %red
\definecolor{cadmiumred}{rgb}{0.89, 0.0, 0.13}          %red
\definecolor{arsenic}{rgb}{0.23, 0.27, 0.29}            %gray
\definecolor{alizarin}{rgb}{0.82, 0.1, 0.26}
\definecolor{lust}{rgb}{0.9, 0.13, 0.13}                %red
\definecolor{lava}{rgb}{0.81, 0.06, 0.13}               %red
\definecolor{harvardcrimson}{rgb}{0.79, 0.0, 0.09}      %red
\definecolor{cadmiumgreen}{rgb}{0.0, 0.42, 0.24}
\definecolor{darkpastelgreen}{rgb}{0.01, 0.75, 0.24}    %lightgreen едренозеленый
\definecolor{darkspringgreen}{rgb}{0.09, 0.45, 0.27}    %emerald green
\definecolor{kellygreen}{rgb}{0.3, 0.73, 0.09}          %kelly green
\definecolor{dartmouthgreen}{rgb}{0.05, 0.5, 0.06}      %green
\definecolor{hookersgreen}{rgb}{0.0, 0.44, 0.0}         %green
\definecolor{amber}{rgb}{1.0, 0.75, 0.0}                %yellow
\definecolor{canaryyellow}{rgb}{1.0, 0.94, 0.0}         %yellow
\definecolor{chromeyellow}{rgb}{1.0, 0.65, 0.0}         %yellow

%=============== Tabularx columns =====================
\newcolumntype{C}{>{\centering\arraybackslash}X}
\newcolumntype{Y}{>{\raggedleft\arraybackslash}X}
\newcolumntype{P}[1]{>{\centering\arraybackslash}p{#1}}

%Some cool feature here, feel free to add some. 
% =======================================================================
% ========= Большой превью в заметках ===================================
% =======================================================================
\setbeamertemplate{note page}%
{%
	{%
			\scriptsize
			\insertvrule{.5\paperheight}{white}
			\vskip-.5\paperheight%
			\nointerlineskip%
			\vbox{
				\hfill\insertslideintonotes{0.5}
				\hskip-1cm\hskip0pt%
				\vskip-0.5\paperheight%
				\nointerlineskip%
				\begin{pgfpicture}{0cm}{0cm}{0cm}{0cm}
					\begin{pgflowlevelscope}{\pgftransformrotate{90}}
						{\pgftransformshift{\pgfpoint{-2cm}{0.2cm}}%
							\pgftext[base,left]{\footnotesize\the\year-\ifnum\month<10\relax0\fi\the\month-\ifnum\day<10\relax0\fi\the\day}}
					\end{pgflowlevelscope}
				\end{pgfpicture}
			}
			\nointerlineskip%
			\vbox to 0.5\paperheight{\vskip0.5em%
				\hbox{\insertshorttitle[width=3cm]}%
				\begin{minipage}{3cm}
					\insertsection \par%
					\insertsubsection%
				\end{minipage}
				\vskip0.5em
				\textbf{Заголовок:} \par
				\hbox{\insertshortframetitle[width=3cm]}%
				\par
				\textbf{Слайд:} \par
				\insertframenumber.\arabic{slidenumber}{}/ \inserttotalframenumber \par
				\vfil}%
		}%
	\vskip-0.025\paperheight%
	\nointerlineskip%
	\insertnote%
}
%================Белый цвет фона в минислайде=================================
\makeatletter
\renewcommand{\insertslideintonotes}[1]{{%
			\begin{pgfpicture}{0cm}{0cm}{#1\paperwidth}{#1\paperheight}
				\begin{pgflowlevelscope}{\pgftransformscale{#1}}%
					\color{white} % меняем тут
					\pgfpathrectangle{\pgfpointorigin}{\pgfpoint{\paperwidth}{\paperheight}}
					\pgfusepath{fill}
					\color{black} %цвет шрифта
					{\pgftransformshift{\pgfpoint{\beamer@origlmargin}{\footheight}}\pgftext[left,bottom]{\copy\beamer@frameboxcopy}}
				\end{pgflowlevelscope}
			\end{pgfpicture}%
		}}
\makeatother
%=====================Конец==================================================

%============================Подсчет количества слайдов с оверлееми===========
\newcounter{slidenumber}

\defbeamertemplate*{footline}{infolines theme frame plus slide}{
	\setcounter{slidenumber}{\insertpagenumber}%
	\addtocounter{slidenumber}{-\insertframestartpage}%
	\addtocounter{slidenumber}{1}%
	\leavevmode%
	\hbox{%
		\begin{beamercolorbox}[wd=.333333\paperwidth,ht=2.25ex,dp=1ex,center]{author in head/foot}%
			\usebeamerfont{author in head/foot}\insertshortauthor~~(\insertshortinstitute)
		\end{beamercolorbox}%
		\begin{beamercolorbox}[wd=.333333\paperwidth,ht=2.25ex,dp=1ex,center]{title in head/foot}%
			\usebeamerfont{title in head/foot}\insertshorttitle
		\end{beamercolorbox}%
		\begin{beamercolorbox}[wd=.333333\paperwidth,ht=2.25ex,dp=1ex,right]{date in head/foot}%
			\usebeamerfont{date in head/foot}\insertshortdate{}\hspace*{2em}
			\insertframenumber.\arabic{slidenumber}{}/ \inserttotalframenumber\hspace*{2ex}
		\end{beamercolorbox}}%
	\vskip0pt%
}
%require to set below
%\setbeamertemplate{footline}[infolines theme frame plus slide]



%====================Буллет в note[item]======================================
\makeatletter
\def\beamer@setupnote{%
	\gdef\beamer@notesactions{%
		\beamer@outsideframenote{%
			\beamer@atbeginnote%
			\beamer@notes%
			\ifx\beamer@noteitems\@empty\else
				\begin{itemize}\itemsep=0pt\parskip=0pt%
					\beamer@noteitems%
				\end{itemize}%
			\fi%
			\beamer@atendnote%
		}%
		\gdef\beamer@notesactions{}%
	}
}

\makeatother
%=================================Конец=====================================



%===========================   Выравнивание по ширине в block  ===========
\addtobeamertemplate{block begin}{}{\justifying}
\renewcommand{\raggedright}{\leftskip=0pt \rightskip=0pt plus 0cm} %global

%============================== Библиография занимает меньше места ======
\setbeamertemplate{bibliography entry title}{}
\setbeamertemplate{bibliography entry location}{}
\setbeamertemplate{bibliography entry note}{}
\setbeamertemplate{bibliography item}{\insertbiblabel}
%===========================================================================

%===========Заменяем библиографию с квадратных скобок на точку
\makeatletter
\renewcommand{\@biblabel}[1]{#1.}
\makeatother
%============================================================

%============ Черный шрифт для библиографии  ==================================
\setbeamercolor{bibliography item}{fg=black}
\setbeamercolor*{bibliography entry title}{fg=black}
\setbeamercolor*{bibliography entry author}{fg=black}
\setbeamercolor*{bibliography entry journal}{fg=black}
\setbeamercolor*{bibliography entry location}{fg=black}
%==============================================================================


%================== Общие макросы и команды ==========================

% просто чтобы не писать много буковов
\newcommand{\tb}[1]{\textsubscript{#1}}
\newcommand{\tp}[1]{\textsuperscript{#1}}

%============ The fullFrameMovie command defenition ==========================
% from early pdfpc sty 
% Package: textpos is required for textblock*
\usepackage[absolute,overlay]{textpos}


% fullFrameMovie
%
% Arguments:
%
%   [optional]: movie-options, seperated by &
%       Supported options: loop, start=N, end=N, autostart
%   Default: autostart&loop
%
%   1. Movie file
%   2. Poster image
%   3. Any text on the slide, or nothing (e.g. {})
%
% Example:
%   \fullFrameMovie[loop&autostart]{apollo17.avi}{apollo17.jpg}{\copyrightText{Apollo 17, NASA}}
%
\newcommand{\fullFrameMovie}[4][autostart&loop]
{
	{
			\setbeamercolor{background canvas}{bg=black}


			% to make this work for both horizontally filled and vertically filled images, we create an absolutely
			% positioned textblock* that we force to be the width of the slide.
			% we then place it at (0,0), and then create a box inside of it to ensure that it's always 95% of the vertical
			% height of the frame.  Once we have created an absolutely positioned and sized box, it doesn't matter what
			% goes inside -- it will always be vertically and horizontally centered
			\frame[plain]
			{
				\begin{textblock*}{\paperwidth}(0\paperwidth,0\paperheight)
					\centering
					\vbox to 0.99\paperheight {
						\vfil{
							\href{run:#2?autostart&#1}{\includegraphics[width=\paperwidth,height=0.95\paperheight,keepaspectratio]{#3}}
						}
						\vfil
					}
				\end{textblock*}
				#4
			}
		}
}

\newcommand{\fullFrameMultimedia}[3]
{
	{
			\setbeamercolor{background canvas}{bg=black}
			\usebackgroundtemplate[default]


			% to make this work for both horizontally filled and vertically filled images, we create an absolutely
			% positioned textblock* that we force to be the width of the slide.
			% we then place it at (0,0), and then create a box inside of it to ensure that it's always 95% of the vertical
			% height of the frame.  Once we have created an absolutely positioned and sized box, it doesn't matter what
			% goes inside -- it will always be vertically and horizontally centered
			\frame[plain]
			{
				\begin{textblock*}{\paperwidth}(0\paperwidth,0\paperheight)
					\centering
					\vbox to 0.99\paperheight {
						\vfil{
							% \href{run:#2?autostart&#1}{\includegraphics[width=\paperwidth,height=0.95\paperheight,keepaspectratio]{#3}}
							\movie[width=\paperwidth,height=0.95\paperheight]{\includegraphics[width=\paperwidth,keepaspectratio]{#2}}{#1}

						}
						\vfil
					}
				\end{textblock*}
				#3
			}
		}
}
% inlineMovie
%
% Arguments:
%
%   [optional]: movie-options, seperated by &
%       Supported options: loop, start=N, end=N, autostart
%   Default: autostart&loop
%
%   1. Movie file
%   2. Poster image
%   3. size command, such as width=\textwidth
%
% Example:
%   \inlineMovie[loop&autostart&start=5&stop=12]{apollo17.avi}{apollo17.jpg}{height=0.7\textheight}
%
\newcommand{\inlineMovie}[4][autostart&loop]
{
	\href{run:#2?#1}{\includegraphics[#4]{#3}}
}


% copyrightText
%
% Produces small text on the right side of the screen, useful for
% stating copyright or other small notes in movies or images
%
% Arguments:
%
%   [optional]: text color
%       Default: white
%
%   1. Text to be displayed
%
% Example:
%   \copyrightText{Full frame image of: Apollo 17, NASA}
%
\newcommand\copyrightText[2][white]{%
	\begin{textblock*}{\paperwidth}(0\paperwidth,.97\paperheight)%
		\hfill\textcolor{#1}{\tiny#2}\hspace{20pt}
	\end{textblock*}
}

% fullFrameImageZoomed
%
% Produces a slide that contains a full frame image.  Scales down the image
% to fit if the aspect ratio of the slide does not match the image.
%
% Arguments:
%
%   [optional]: color of text on page
%       Default: white
%
%   1. Path to image file
%   2. Any additional content on the frame
%
% Example:
%   \fullFrameImageZoomed{apollo17.jpg}{\copyrightText{Full frame image of: Apollo 17, NASA}}
%
\newcommand{\fullFrameImage}[3][white]
{
	{
			\setbeamercolor{normal text}{bg=black,fg=#1}


			% to make this work for both horizontally filled and vertically filled images, we create an absolutely
			% positioned textblock* that we force to be the width of the slide.
			% we then place it at (0,0), and then create a box inside of it to ensure that it's always 95% of the vertical
			% height of the frame.  Once we have created an absolutely positioned and sized box, it doesn't matter what
			% goes inside -- it will always be vertically and horizontally centered
			\frame
			{
				\begin{textblock*}{\paperwidth}(0\paperwidth,0\paperheight)
					\centering
					\vbox to 0.95\paperheight {
						\vfil{
							\includegraphics[width=\paperwidth,height=0.95\paperheight,keepaspectratio]{#2}
						}
						\vfil
					}
				\end{textblock*}
				#3
			}
		}
}

% fullFrameImageZoomed
%
% Produces a slide that contains a full frame image.  If the aspect ratio
% of the image does not match the slide, it crops the image.
%
% Arguments:
%
%   [optional]: color of text on page
%       Default: black
%
%   1. Path to image file
%   2. Any additional content on the frame
%
% Example:
%   \fullFrameImageZoomed{apollo17.jpg}{\copyrightText{Full frame image of: Apollo 17, NASA}}
%
\newcommand{\fullFrameImageZoomed}[3][black]
{
	{
			\usebackgroundtemplate{\includegraphics[height=\paperheight]{#2}}
			\setbeamercolor{normal text}{bg=black,fg=#1}
			\frame
			{
				#3
			}
		}
}

% Уменьшает вертикальное расстояние в equation
\makeatletter
\g@addto@macro\normalsize{%
	\setlength\belowdisplayskip{-0pt}
	\setlength\abovedisplayskip{-1pt}
}
\makeatother


% handdrawning decorations
% calc, decorations.pathmorphing, patterns,
%TODO write help
\pgfdeclaredecoration{penciline}{initial}{
	\state{initial}[width=+\pgfdecoratedinputsegmentremainingdistance,
		auto corner on length=1mm,]{
		\pgfpathcurveto%
		{% From
			\pgfqpoint{\pgfdecoratedinputsegmentremainingdistance}
			{\pgfdecorationsegmentamplitude}
		}
		{%  Control 1
			\pgfmathrand
			\pgfpointadd{\pgfqpoint{\pgfdecoratedinputsegmentremainingdistance}{0pt}}
			{\pgfqpoint{-\pgfdecorationsegmentaspect
					\pgfdecoratedinputsegmentremainingdistance}%
				{\pgfmathresult\pgfdecorationsegmentamplitude}
			}
		}
		{%TO 
			\pgfpointadd{\pgfpointdecoratedinputsegmentlast}{\pgfpoint{1pt}{1pt}}
		}
	}
	\state{final}{}
}


% Уменьшает отступы в листинге
\makeatletter
\def\nbhline{%
	\noalign{\ifnum0=`}\fi
		\penalty\@M%
		\futurelet\@let@token\LT@@nobreakhline}
	\def\LT@@nobreakhline{%
		\ifx\@let@token\hline
			\global\let\@gtempa\@gobble%
			\gdef\LT@sep{\penalty\@M\vskip\doublerulesep}% <-- change here
		\else
			\global\let\@gtempa\@empty%
			\gdef\LT@sep{\penalty\@M\vskip-\arrayrulewidth}% <-- change here
		\fi
		\ifnum0=`{\fi}%
	\multispan\LT@cols%
	\unskip\leaders\hrule\@height\arrayrulewidth\hfill\cr
	\noalign{\LT@sep}%
	\multispan\LT@cols%
	\unskip\leaders\hrule\@height\arrayrulewidth\hfill\cr
	\noalign{\penalty\@M}%
	\@gtempa}
\makeatother






\def\preambleloaded{Precompiled preamble loaded.}
