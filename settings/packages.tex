%Here some packets for beamer presentations. Put needed packages here
%\documentclass[pdf,hyperref={unicode},xcolor={x11names},svgnames]{beamer} % 10.2022 vimtex fail to detect master tex file, so it moved to main file
\usepackage[T2A]{fontenc}
\usepackage[utf8]{inputenc}
\usepackage[english,russian]{babel}
\usepackage{amssymb,
	amsfonts,
	array,
	wasysym,
	%fixltx2e,
	amsmath,
	empheq,         % group equation
	mathtext,
	gensymb,
	textcomp,        % obsolate
	% cite,            % removed for biblatex
	multirow,
	enumerate,
	float,
	icomma,
	xcolor,
	fancybox,
	soulutf8,
	csquotes,
	minted,
	multimedia, %\movie[width=200pt, height=150pt]{example video}{MRauma.mpg}
	booktabs,
	indentfirst,
	rotating,
	fontawesome, % занчки ютуба и стима
	tabularx,
	tabulary,
	pdfpc,      % presentation helper
	pgf
}

%\usefonttheme[onlymath]{serif} %cm math font  
\usefonttheme{professionalfonts}

\usepackage[per-mode=symbol]{siunitx} % Provides the \SI{}{} command for typesetting SI units, набор значений единиц измерения

% Позволяет использовать конструкцию где
\usepackage{eqexpl}

\usepackage[most]{tcolorbox}
\usepackage{colortbl}
%добавить в includegraphics[width=...frame]{image}

\usepackage{graphicx} %хотим вставлять в рисунки?
\graphicspath{{images/}{inc/}}%путь к рисункам

\usepackage{enumitem}
\setitemize{label=\usebeamerfont*{itemize item}%
	\usebeamercolor[fg]{itemize item}
	\usebeamertemplate{itemize item}, topsep=0pt}
\setlist[enumerate]{label*=\arabic*. , leftmargin=15pt }


\usepackage[
	parentracker=true,
	backend=biber,
	hyperref=auto,
	language=auto,
	autolang=other, % иногда даетм много места в скобках
	citestyle=gost-numeric,
	defernumbers=true,
	bibstyle=gost-numeric,
]
{biblatex}

\DeclareSourcemap{
	\maps[datatype=bibtex]{
		\map{
			\step[fieldset=language,fieldvalue=english] % fill empty languge field
			\step[fieldsource=language,fieldtarget=langid, final] % fill langid with language value
			%\step[fieldset=langid, fieldvalue={,}, append]
			%\step[fieldset=keywords, origfieldval, append]
		}
	}
}
\addbibresource{biblio/jabref.bib}
\addbibresource{biblio/activ.bib}
\addbibresource{biblio/my.bib}
\addbibresource{biblio/ochistka.bib}
%\printbibliography

% \setbeamercolor{bibliography entry title}{fg=blue!50!cyan}
% \setbeamercolor{bibliography entry author}{fg=violet}
% \setbeamercolor{bibliography entry location}{fg=green}
\setbeamercolor{bibliography entry note}{fg=black}

%============Химия=========================================================================
\usepackage[version=4]{mhchem}
\usepackage{chemfig} % рисование структурных формул в химии <<настоящий ад и вынос мозга>>
\makeatletter %использование mhchem для строчных атомов в структурных формулах
\def\CF@node@content{%
	\expandafter\expandafter\expandafter
	\printatom\expandafter\expandafter\expandafter
	{\csname atom@\number\CF@cnt@atomnumber\endcsname}%
	\ensuremath{\CF@node@strut}%
}
\makeatother
\setchemfig{
	double bond sep=0.35700 em,
	atom sep=1.78500 em,
	bond offset=0.18265 em,
	bond style={line width =0.06642 em}
}
%============================Позволяет рисовать полимеры===========
\newcommand\setpolymerdelim[2]{\def\delimleft{#1}\def\delimright{#2}}
\def\makebraces[#1,#2]#3#4#5{%
	\edef\delimhalfdim{\the\dimexpr(#1+#2)/2}%
	\edef\delimvshift{\the\dimexpr(#1-#2)/2}%
	\chemmove{%
		\node[at=(#4),yshift=(\delimvshift)]
		{$\left\delimleft\vrule height\delimhalfdim depth\delimhalfdim
				width0pt\right.$};%
		\node[at=(#5),yshift=(\delimvshift)]
		{$\left.\vrule height\delimhalfdim depth\delimhalfdim
				width0pt\right\delimright_{\rlap{$\scriptstyle#3$}}$};}}
%=============Пример==================================
%%\setpolymerdelim()% выбор типа скобок
%%Polyéthylène:
%%\chemfig{\vphantom{CH_2}%используется для поднятия связи на нормальное расстояние от базовой линии
%%	-[@{op,.75}]CH_2-CH_2-[@{cl,0.25}]}
%%\makebraces[5pt,5pt]{\!\!n=13}{op}{cl}
%===============конец химии=======================


%=======Заметки в презентации====
\usepackage{pgfpages}
%\setbeameroption{show notes on second screen=bottom} % расположение заметок, закоментить для отключения 
\setbeamerfont{note page}{size=\footnotesize} %шрифт
%\note{My text} % example
%\note<2> [item]{My enumerate} % example
%===================
%=================  Минд мапы  ====================================
\usepackage{tikz}
\usetikzlibrary{
	mindmap,
	positioning,
	arrows,
	shapes,
	shapes.geometric,
	shapes.callouts,
	shapes.arrows,
	calc,
	decorations.pathmorphing,
	patterns,
	shadows}
%============  Набор физических величин  ==================================
\usepackage{siunitx} % Provides the \SI{}{} command for typesetting SI units, набор значений единиц измерения
\sisetup{range-phrase=--,range-units = single,locale = DE} % no-russian :((

%============  Выравнивание в тексте   ======================
\usepackage{ragged2e}
\justifying

\usepackage[framemethod=TikZ]{mdframed} % красивые боксы вокруг слов


\usepackage[export]{adjustbox}% рамка вокруг рисунков

\makeatletter
%\patchcmd{\@listI}{\itemsep3\p@}{\itemsep0em}{}{}
\renewcommand{\@listI}{%
	\leftmargin=10pt
	\rightmargin=0pt
	\labelsep=5pt
	\labelwidth=20pt
	\itemindent=0pt
	\listparindent=0pt
	\topsep=8pt plus 2pt minus 4pt
	\partopsep=2pt plus 1pt minus 1pt
	\parsep=0pt plus 1pt
	\itemsep=\parsep}
\makeatother

\usepackage{listings}                % пакет для набора исходных текстов программ
\usepackage{lstautogobble}           % доп пакет для игнорирования начальных пробелов
% \usepackage{listingsutf8}           % пакет для набора исходных текстов программ
\usepackage{xcolor}                  % пакет для цвета, включен здесь если ранее не включен

\usepackage[outline]{contour}
\contourlength{1.2pt}

%=================  Конец пакетов  =============================================
% Добавляет содержание перед секцией
\AtBeginSection[]
{\begin{frame}
		\frametitle{Содержание}
		\tableofcontents[currentsection]
	\end{frame} }






%====================  Colors  ===========================================================
\definecolor{airforceblue}{rgb}{0.36, 0.54, 0.66}       %blue
\definecolor{aliceblue}{rgb}{0.94, 0.97, 1.0}           %white
\definecolor{darkslateblue}{rgb}{0.28, 0.24, 0.55}      %blue
\definecolor{antiquefuchsia}{rgb}{0.57, 0.36, 0.51}     %purple
\definecolor{darkraspberry}{rgb}{0.53, 0.15, 0.34}      %purple

\definecolor{debianred}{rgb}{0.84, 0.04, 0.33}          %red
\definecolor{darkred}{rgb}{0.55, 0.0, 0.0}              %red
\definecolor{cadmiumred}{rgb}{0.89, 0.0, 0.13}          %red
\definecolor{arsenic}{rgb}{0.23, 0.27, 0.29}            %gray
\definecolor{alizarin}{rgb}{0.82, 0.1, 0.26}
\definecolor{lust}{rgb}{0.9, 0.13, 0.13}                %red
\definecolor{lava}{rgb}{0.81, 0.06, 0.13}               %red
\definecolor{harvardcrimson}{rgb}{0.79, 0.0, 0.09}      %red
\definecolor{cadmiumgreen}{rgb}{0.0, 0.42, 0.24}
\definecolor{darkpastelgreen}{rgb}{0.01, 0.75, 0.24}    %lightgreen едренозеленый
\definecolor{darkspringgreen}{rgb}{0.09, 0.45, 0.27}    %emerald green
\definecolor{kellygreen}{rgb}{0.3, 0.73, 0.09}          %kelly green
\definecolor{dartmouthgreen}{rgb}{0.05, 0.5, 0.06}      %green
\definecolor{hookersgreen}{rgb}{0.0, 0.44, 0.0}         %green
\definecolor{amber}{rgb}{1.0, 0.75, 0.0}                %yellow
\definecolor{canaryyellow}{rgb}{1.0, 0.94, 0.0}         %yellow
\definecolor{chromeyellow}{rgb}{1.0, 0.65, 0.0}         %yellow

%=============== Tabularx columns =====================
\newcolumntype{C}{>{\centering\arraybackslash}X}
\newcolumntype{Y}{>{\raggedleft\arraybackslash}X}
\newcolumntype{P}[1]{>{\centering\arraybackslash}p{#1}}

../../../settings/extensions.tex



\def\preambleloaded{Precompiled preamble loaded.}
