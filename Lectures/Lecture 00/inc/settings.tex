
\usetheme[block=fill,progressbar=frametitle,background=light]{moloch}
% Цвета ИТМО
\usecolortheme{moloch-itmo}

% Тут включается отображение заметок.
\setbeameroption{show notes on second screen=top} % расположение заметок 

\setbeamercovered{transparent} % полупрозрачность для \uncover

\hypersetup{
	colorlinks=true,
	linkcolor=black, % дефолтный цвет из темы
	filecolor=magenta,
	urlcolor=cyan,
	pdftitle={Презентация ВКР},
	pdfpagemode=FullScreen,
}
%=====================================ЛОГО======================================

% Логотип на Титульной странице
% Тут логотип отправлен вправо коммандой \hfill
% \titlegraphic{\hfill\includegraphics[width=0.3\textwidth]{./inc/LogoSPbFTU-2.pdf}}
\titlegraphic{\includegraphics[width=0.2\textwidth]{itmo/logo_basic_russian_white.pdf}}
% Логотип на остальных страницах
% \logo{
% 	\begin{minipage}{5em}
% 		\begin{block}{}
% 			\centering
% 			\bfseries
% 			Powered by \\ {\small\textrm\LaTeX{}}  \\  and LAB\,60
% 		\end{block}
% 	\end{minipage}\hspace{1em}
% }


% Фоновый рисунок обычный
% \setbeamertemplate{background canvas}{\includegraphics[width=\paperwidth]{background}}
% Фоновый рисунок тайловый
% \setbeamertemplate{background canvas}{%
% 	\vbox{%
% 		\foreach\y in {1,...,2}{%
% 				\foreach\x in {1,...,2}{%
% 						\includegraphics[width=.5\paperwidth]{background}%
% 					}%
% 				\linebreak
% 			}%
% 	}%
% }
% Настройка слайда с названием секции
\AtBeginSection{
	\begingroup
	\setbeamertemplate{background canvas}[default]
	% \setbeamercolor{background canvas}{bg=grayFTU} % используется цвет из темы FTU
	\logo{}
	\frame{\sectionpage}
	\endgroup
}

% Использовать блоки с тенью
\setbeamertemplate{blocks}[rounded][shadow=true]
% Визуальный эффект смены слайдов
\addtobeamertemplate{background canvas}{\transfade[duration=2]}{}

%======================== Изменение ЦВЕТА темы ======================================
%\setbeamercolor{alerted text}{fg=orange}
%\setbeamercolor{background canvas}{bg=white}
%================цвет блока для всего документа==============
%\setbeamercolor{block body alerted}{bg=normal text.bg!90!black}
%\setbeamercolor{block body}{bg=normal text.bg!90!black}
%\setbeamercolor{block body example}{bg=normal text.bg!90!black}
%\setbeamercolor{block title alerted}{use={normal text,alerted text},fg=alerted text.fg!75!normal text.fg,bg=normal text.bg!75!black}
%\setbeamercolor{block title}{bg=blue}
%\setbeamercolor{block title example}{use={normal text,example text},fg=example text.fg!75!normal text.fg,bg=normal text.bg!75!black}
%\setbeamercolor{fine separation line}{}
%\setbeamercolor{frametitle}{fg=brown}
%\setbeamercolor{item projected}{fg=black}
%\setbeamercolor{normal text}{bg=black,fg=yellow}
%\setbeamercolor{palette sidebar primary}{use=normal text,fg=normal text.fg}
%\setbeamercolor{palette sidebar quaternary}{use=structure,fg=structure.fg}
%\setbeamercolor{palette sidebar secondary}{use=structure,fg=structure.fg}
%\setbeamercolor{palette sidebar tertiary}{use=normal text,fg=normal text.fg}
%\setbeamercolor{section in sidebar}{fg=brown}
%\setbeamercolor{section in sidebar shaded}{fg=grey}
%\setbeamercolor{separation line}{}
%\setbeamercolor{sidebar}{bg=red}
%\setbeamercolor{sidebar}{parent=palette primary}
%\setbeamercolor{structure}{bg=black, fg=green}
%\setbeamercolor{subsection in sidebar}{fg=brown}
%\setbeamercolor{subsection in sidebar shaded}{fg=grey}
%\setbeamercolor{title}{fg=brown}
%\setbeamercolor{titlelike}{fg=brown}
%===============Конец описания цветов=======================


%=============================БЛОКИ==========================================
%цвет блока для всего документа
%\setbeamercolor{block title}{fg=orange!20!black,bg=green}%bg=background, fg= foreground
%\setbeamercolor{block body}{bg=green!10,fg=black}%bg=background, fg= foreground

%создаем специальный блок другого цвета theorem, exampleblock, alertblock уже есть!
\newenvironment<>{problock}[1]{%
	\begin{actionenv}#2%
		\def\insertblocktitle{#1}%
		\par%
		\mode<presentation>{%
			\setbeamercolor{block title}{fg=white,bg=orange!20!black}
			\setbeamercolor{block body}{fg=black,bg=yellow!20}
			\setbeamercolor{itemize item}{fg=orange!20!black}
			\setbeamertemplate{itemize item}[triangle]
		}%
		\usebeamertemplate{block begin}}
		{\par\usebeamertemplate{block end}\end{actionenv}}

%============TIKZ Styles=========================
%
% \tikzstyle{subst} = [rounded rectangle,
% thick,
% inner sep=5pt,
% minimum size=1cm,
% draw=red!50!black!50,
% top color=white,
% bottom color=red!50!black!50,
% font=\itshape,
% drop shadow]
% \tikzstyle{process} = [ellipse,
% thick,
% minimum size=1cm,
% draw=blue!50!black!50,
% top color=white,
% bottom color=blue!50!black!20,
% drop shadow]
%============END of TIKZ Styles=====================

% Настройка tcolorbox
\tcbuselibrary{breakable}
\tcbuselibrary{skins}

\newtcolorbox{mybox}[2][]%
{%
	attach boxed title to top center
		= {yshift=-8pt},
	%colback      = blue!5!white,
	%colframe     = blue!75!black,
	fonttitle    = \bfseries,
	%opacityback=0.8,
	%opacitybacktitle=0.75,
	%colbacktitle = blue!85!black,
	breakable,
	left=0mm,
	right=0mm,
	%left skip=-2mm,
	%right skip=-2mm,
	title        = #2,#1,
	skin=enhanced
}
\newtcolorbox{process}[2][]%
{%
	attach boxed title to top center
		= {yshift=-8pt},
	%colback      = blue!5!white,
	%colframe     = blue!75!black,
	fonttitle    = \bfseries,
	%opacityback=0.8,
	%opacitybacktitle=0.75,
	colbacktitle = orange,
	breakable,
	left=0mm,
	right=0mm,
	%left skip=-2mm,
	%right skip=-2mm,
	title        = #2,#1,
	skin=enhanced,
}

\newtcolorbox{autotherm}[2][]%
{%
	attach boxed title to top center
		= {yshift=-8pt},
	%colback      = blue!5!white,
	%colframe     = blue!75!black,
	fonttitle    = white,
	%opacityback=0.8,
	%opacitybacktitle=0.75,
	colupper=white,
	colback = orange,
	breakable,
	left=0mm,
	right=0mm,
	%left skip=-2mm,
	%right skip=-2mm,
	title        = #2,#1,
	skin=enhanced,
}
\newtcolorbox{timeline}[2][]%
{%
	attach boxed title to top center
		= {yshift=-8pt},
	%colback      = blue!5!white,
	%colframe     = blue!75!black,
	%fonttitle    = white,
	%opacityback=0.8,
	%opacitybacktitle=0.75,
	%colupper=white,
	%colback = white,
	colbacktitle = white,
	breakable,
	left=0mm,
	right=0mm,
	%left skip=-2mm,
	%right skip=-2mm,
	title        = #2,#1,
	skin=enhanced,
}
\newtcbox{\xmybox}{
	on line,
	arc=7pt,
	before upper={\rule[-3pt]{0pt}{5pt}},
	boxrule=1pt,
	boxsep=0pt,
	left=2pt,
	right=2pt,
	top=3pt,
	bottom=2pt
}

% Новый тип колонки для примера
% \newcolumntype{Y}{>{\raggedleft\arraybackslash}X}

\tcbset{tab1/.style={fonttitle=\bfseries\large,fontupper=\normalsize\sffamily,
			colback=green!10!white,colframe=green!75!black,colbacktitle=green!40!white,
			coltitle=black,center title,freelance,frame code={
					\foreach \n in {north east,north west,south east,south west}
						{\path [fill=darkgreen] (interior.\n) circle (3mm); };},}}

\tcbset{tab2/.style={enhanced,fonttitle=\bfseries,fontupper=\normalsize\sffamily,
			colback=green!10!white,colframe=green!50!black,colbacktitle=green!40!white,
			coltitle=black,center title}}


%TODO
% \useinnertheme{tcolorbox}

% Настройка листинга для кода

\lstset{ %
	language={[latex]TeX},          % язык программирования
	inputencoding=utf8,             % кодировка
	autogobble=true,                % убирает пробелы в начале строки
	basicstyle=\ttfamily\footnotesize,       % the size of the fonts that are used for the code
	numbers=left,                   % where to put the line-numbers
	numberstyle=\tiny\color{gray},  % the style that is used for the line-numbers
	stepnumber=1,                   % the step between two line-numbers. If it's 1, each line
	% will be numbered
	numbersep=0pt,                  % how far the line-numbers are from the code
	% backgroundcolor=\color{white},  % choose the background color. You must add \usepackage{color}
	showspaces=false,               % show spaces adding particular underscores
	showstringspaces=false,         % underline spaces within strings
	showtabs=false,                 % show tabs within strings adding particular underscores
	% frame=single,                   % adds a frame around the code
	rulecolor=\color{black},        % if not set, the frame-color may be changed on line-breaks within not-black text (e.g. commens (green here))
	tabsize=2,                      % sets default tabsize to 2 spaces
	captionpos=b,                   % sets the caption-position to bottom
	title=\lstname,                 % show the filename of files included with \lstinputlisting;
	% also try caption instead of title
	keywordstyle=\color{green},       % keyword style
	commentstyle=\color{blue},      % comment style
	stringstyle=\color{gray},      % string literal style
	escapeinside={\%*}{*)},         % if you want to add a comment within your code
	extendedchars=\true,            % поддержка расширенных символов
	keepspaces = true               %!!!! пробелы в комментариях
	texcl=true,
	breaklines=true,                % разрывать строку
	breakatwhitespace=true,         % разрыв по пробелам
	aboveskip=-0.25 \baselineskip,   % верхний интервал
	belowskip=-1.5 \baselineskip,   % нижный интервал
	morekeywords={*,note,... }      % if you want to add more keywords to the set
}
\renewcommand\lstlistingname{Листинг}


\useinnertheme[
	rounded,
	% shadow
]{tcolorbox}

% \tcbsetforeverylayer{
% 	borderline={1pt}{0pt}{greenFTU!40!black}
% }

