%&packages
% !TeX spellcheck = ru_RU
\documentclass[pdf,hyperref={unicode},xcolor={x11names},svgnames]{beamer} % 10.2022 vimtex fail to detect master tex file, so it moved to main file


\def\ifundefined#1{\expandafter\ifx\csname#1\endcsname\relax} \ifundefined{preambleloaded}
\typeout{PRECOMILED PREAMBLE NOT LOADED}%Here some packets for beamer presentations. Put needed packages here
%\documentclass[pdf,hyperref={unicode},xcolor={x11names},svgnames]{beamer} % 10.2022 vimtex fail to detect master tex file, so it moved to main file
\usepackage[T2A]{fontenc}
\usepackage[utf8]{inputenc}
\usepackage[english,russian]{babel}
\usepackage{amssymb,
	amsfonts,
	array,
	wasysym,
	%fixltx2e,
	amsmath,
	empheq,         % group equation
	mathtext,
	gensymb,
	textcomp,        % obsolate
	% cite,            % removed for biblatex
	multirow,
	enumerate,
	float,
	icomma,
	xcolor,
	fancybox,
	soulutf8,
	csquotes,
	minted,
	multimedia, %\movie[width=200pt, height=150pt]{example video}{MRauma.mpg}
	booktabs,
	indentfirst,
	rotating,
	fontawesome, % занчки ютуба и стима
	tabularx,
	tabulary,
	pdfpc,      % presentation helper
	pgf
}

%\usefonttheme[onlymath]{serif} %cm math font  
\usefonttheme{professionalfonts}

\usepackage[per-mode=symbol]{siunitx} % Provides the \SI{}{} command for typesetting SI units, набор значений единиц измерения

% Позволяет использовать конструкцию где
\usepackage{eqexpl}

\usepackage[most]{tcolorbox}
\usepackage{colortbl}
%добавить в includegraphics[width=...frame]{image}

\usepackage{graphicx} %хотим вставлять в рисунки?
\graphicspath{{images/}{inc/}}%путь к рисункам

\usepackage{enumitem}
\setitemize{label=\usebeamerfont*{itemize item}%
	\usebeamercolor[fg]{itemize item}
	\usebeamertemplate{itemize item}, topsep=0pt}
\setlist[enumerate]{label*=\arabic*. , leftmargin=15pt }


\usepackage[
	parentracker=true,
	backend=biber,
	hyperref=auto,
	language=auto,
	autolang=other, % иногда даетм много места в скобках
	citestyle=gost-numeric,
	defernumbers=true,
	bibstyle=gost-numeric,
]
{biblatex}

\DeclareSourcemap{
	\maps[datatype=bibtex]{
		\map{
			\step[fieldset=language,fieldvalue=english] % fill empty languge field
			\step[fieldsource=language,fieldtarget=langid, final] % fill langid with language value
			%\step[fieldset=langid, fieldvalue={,}, append]
			%\step[fieldset=keywords, origfieldval, append]
		}
	}
}
\addbibresource{biblio/jabref.bib}
\addbibresource{biblio/activ.bib}
\addbibresource{biblio/my.bib}
\addbibresource{biblio/ochistka.bib}
%\printbibliography

% \setbeamercolor{bibliography entry title}{fg=blue!50!cyan}
% \setbeamercolor{bibliography entry author}{fg=violet}
% \setbeamercolor{bibliography entry location}{fg=green}
\setbeamercolor{bibliography entry note}{fg=black}

%============Химия=========================================================================
\usepackage[version=4]{mhchem}
\usepackage{chemfig} % рисование структурных формул в химии <<настоящий ад и вынос мозга>>
\makeatletter %использование mhchem для строчных атомов в структурных формулах
\def\CF@node@content{%
	\expandafter\expandafter\expandafter
	\printatom\expandafter\expandafter\expandafter
	{\csname atom@\number\CF@cnt@atomnumber\endcsname}%
	\ensuremath{\CF@node@strut}%
}
\makeatother
\setchemfig{
	double bond sep=0.35700 em,
	atom sep=1.78500 em,
	bond offset=0.18265 em,
	bond style={line width =0.06642 em}
}
%============================Позволяет рисовать полимеры===========
\newcommand\setpolymerdelim[2]{\def\delimleft{#1}\def\delimright{#2}}
\def\makebraces[#1,#2]#3#4#5{%
	\edef\delimhalfdim{\the\dimexpr(#1+#2)/2}%
	\edef\delimvshift{\the\dimexpr(#1-#2)/2}%
	\chemmove{%
		\node[at=(#4),yshift=(\delimvshift)]
		{$\left\delimleft\vrule height\delimhalfdim depth\delimhalfdim
				width0pt\right.$};%
		\node[at=(#5),yshift=(\delimvshift)]
		{$\left.\vrule height\delimhalfdim depth\delimhalfdim
				width0pt\right\delimright_{\rlap{$\scriptstyle#3$}}$};}}
%=============Пример==================================
%%\setpolymerdelim()% выбор типа скобок
%%Polyéthylène:
%%\chemfig{\vphantom{CH_2}%используется для поднятия связи на нормальное расстояние от базовой линии
%%	-[@{op,.75}]CH_2-CH_2-[@{cl,0.25}]}
%%\makebraces[5pt,5pt]{\!\!n=13}{op}{cl}
%===============конец химии=======================


%=======Заметки в презентации====
\usepackage{pgfpages}
%\setbeameroption{show notes on second screen=bottom} % расположение заметок, закоментить для отключения 
\setbeamerfont{note page}{size=\footnotesize} %шрифт
%\note{My text} % example
%\note<2> [item]{My enumerate} % example
%===================
%=================  Минд мапы  ====================================
\usepackage{tikz}
\usetikzlibrary{
	mindmap,
	positioning,
	arrows,
	shapes,
	shapes.geometric,
	shapes.callouts,
	shapes.arrows,
	calc,
	decorations.pathmorphing,
	patterns,
	shadows}
%============  Набор физических величин  ==================================
\usepackage{siunitx} % Provides the \SI{}{} command for typesetting SI units, набор значений единиц измерения
\sisetup{range-phrase=--,range-units = single,locale = DE} % no-russian :((

%============  Выравнивание в тексте   ======================
\usepackage{ragged2e}
\justifying

\usepackage[framemethod=TikZ]{mdframed} % красивые боксы вокруг слов


\usepackage[export]{adjustbox}% рамка вокруг рисунков

\makeatletter
%\patchcmd{\@listI}{\itemsep3\p@}{\itemsep0em}{}{}
\renewcommand{\@listI}{%
	\leftmargin=10pt
	\rightmargin=0pt
	\labelsep=5pt
	\labelwidth=20pt
	\itemindent=0pt
	\listparindent=0pt
	\topsep=8pt plus 2pt minus 4pt
	\partopsep=2pt plus 1pt minus 1pt
	\parsep=0pt plus 1pt
	\itemsep=\parsep}
\makeatother

\usepackage{listings}                % пакет для набора исходных текстов программ
\usepackage{lstautogobble}           % доп пакет для игнорирования начальных пробелов
% \usepackage{listingsutf8}           % пакет для набора исходных текстов программ
\usepackage{xcolor}                  % пакет для цвета, включен здесь если ранее не включен

\usepackage[outline]{contour}
\contourlength{1.2pt}

%=================  Конец пакетов  =============================================
% Добавляет содержание перед секцией
\AtBeginSection[]
{\begin{frame}
		\frametitle{Содержание}
		\tableofcontents[currentsection]
	\end{frame} }






%====================  Colors  ===========================================================
\definecolor{airforceblue}{rgb}{0.36, 0.54, 0.66}       %blue
\definecolor{aliceblue}{rgb}{0.94, 0.97, 1.0}           %white
\definecolor{darkslateblue}{rgb}{0.28, 0.24, 0.55}      %blue
\definecolor{antiquefuchsia}{rgb}{0.57, 0.36, 0.51}     %purple
\definecolor{darkraspberry}{rgb}{0.53, 0.15, 0.34}      %purple

\definecolor{debianred}{rgb}{0.84, 0.04, 0.33}          %red
\definecolor{darkred}{rgb}{0.55, 0.0, 0.0}              %red
\definecolor{cadmiumred}{rgb}{0.89, 0.0, 0.13}          %red
\definecolor{arsenic}{rgb}{0.23, 0.27, 0.29}            %gray
\definecolor{alizarin}{rgb}{0.82, 0.1, 0.26}
\definecolor{lust}{rgb}{0.9, 0.13, 0.13}                %red
\definecolor{lava}{rgb}{0.81, 0.06, 0.13}               %red
\definecolor{harvardcrimson}{rgb}{0.79, 0.0, 0.09}      %red
\definecolor{cadmiumgreen}{rgb}{0.0, 0.42, 0.24}
\definecolor{darkpastelgreen}{rgb}{0.01, 0.75, 0.24}    %lightgreen едренозеленый
\definecolor{darkspringgreen}{rgb}{0.09, 0.45, 0.27}    %emerald green
\definecolor{kellygreen}{rgb}{0.3, 0.73, 0.09}          %kelly green
\definecolor{dartmouthgreen}{rgb}{0.05, 0.5, 0.06}      %green
\definecolor{hookersgreen}{rgb}{0.0, 0.44, 0.0}         %green
\definecolor{amber}{rgb}{1.0, 0.75, 0.0}                %yellow
\definecolor{canaryyellow}{rgb}{1.0, 0.94, 0.0}         %yellow
\definecolor{chromeyellow}{rgb}{1.0, 0.65, 0.0}         %yellow

%=============== Tabularx columns =====================
\newcolumntype{C}{>{\centering\arraybackslash}X}
\newcolumntype{Y}{>{\raggedleft\arraybackslash}X}
\newcolumntype{P}[1]{>{\centering\arraybackslash}p{#1}}

%Some cool feature here, feel free to add some. 
% =======================================================================
% ========= Большой превью в заметках ===================================
% =======================================================================
\setbeamertemplate{note page}%
{%
	{%
			\scriptsize
			\insertvrule{.5\paperheight}{white}
			\vskip-.5\paperheight%
			\nointerlineskip%
			\vbox{
				\hfill\insertslideintonotes{0.5}
				\hskip-1cm\hskip0pt%
				\vskip-0.5\paperheight%
				\nointerlineskip%
				\begin{pgfpicture}{0cm}{0cm}{0cm}{0cm}
					\begin{pgflowlevelscope}{\pgftransformrotate{90}}
						{\pgftransformshift{\pgfpoint{-2cm}{0.2cm}}%
							\pgftext[base,left]{\footnotesize\the\year-\ifnum\month<10\relax0\fi\the\month-\ifnum\day<10\relax0\fi\the\day}}
					\end{pgflowlevelscope}
				\end{pgfpicture}
			}
			\nointerlineskip%
			\vbox to 0.5\paperheight{\vskip0.5em%
				\hbox{\insertshorttitle[width=3cm]}%
				\begin{minipage}{3cm}
					\insertsection \par%
					\insertsubsection%
				\end{minipage}
				\vskip0.5em
				\textbf{Заголовок:} \par
				\hbox{\insertshortframetitle[width=3cm]}%
				\par
				\textbf{Слайд:} \par
				\insertframenumber.\arabic{slidenumber}{}/ \inserttotalframenumber \par
				\vfil}%
		}%
	\vskip-0.025\paperheight%
	\nointerlineskip%
	\insertnote%
}
%================Белый цвет фона в минислайде=================================
\makeatletter
\renewcommand{\insertslideintonotes}[1]{{%
			\begin{pgfpicture}{0cm}{0cm}{#1\paperwidth}{#1\paperheight}
				\begin{pgflowlevelscope}{\pgftransformscale{#1}}%
					\color{white} % меняем тут
					\pgfpathrectangle{\pgfpointorigin}{\pgfpoint{\paperwidth}{\paperheight}}
					\pgfusepath{fill}
					\color{black} %цвет шрифта
					{\pgftransformshift{\pgfpoint{\beamer@origlmargin}{\footheight}}\pgftext[left,bottom]{\copy\beamer@frameboxcopy}}
				\end{pgflowlevelscope}
			\end{pgfpicture}%
		}}
\makeatother
%=====================Конец==================================================

%============================Подсчет количества слайдов с оверлееми===========
\newcounter{slidenumber}

\defbeamertemplate*{footline}{infolines theme frame plus slide}{
	\setcounter{slidenumber}{\insertpagenumber}%
	\addtocounter{slidenumber}{-\insertframestartpage}%
	\addtocounter{slidenumber}{1}%
	\leavevmode%
	\hbox{%
		\begin{beamercolorbox}[wd=.333333\paperwidth,ht=2.25ex,dp=1ex,center]{author in head/foot}%
			\usebeamerfont{author in head/foot}\insertshortauthor~~(\insertshortinstitute)
		\end{beamercolorbox}%
		\begin{beamercolorbox}[wd=.333333\paperwidth,ht=2.25ex,dp=1ex,center]{title in head/foot}%
			\usebeamerfont{title in head/foot}\insertshorttitle
		\end{beamercolorbox}%
		\begin{beamercolorbox}[wd=.333333\paperwidth,ht=2.25ex,dp=1ex,right]{date in head/foot}%
			\usebeamerfont{date in head/foot}\insertshortdate{}\hspace*{2em}
			\insertframenumber.\arabic{slidenumber}{}/ \inserttotalframenumber\hspace*{2ex}
		\end{beamercolorbox}}%
	\vskip0pt%
}
%require to set below
%\setbeamertemplate{footline}[infolines theme frame plus slide]



%====================Буллет в note[item]======================================
\makeatletter
\def\beamer@setupnote{%
	\gdef\beamer@notesactions{%
		\beamer@outsideframenote{%
			\beamer@atbeginnote%
			\beamer@notes%
			\ifx\beamer@noteitems\@empty\else
				\begin{itemize}\itemsep=0pt\parskip=0pt%
					\beamer@noteitems%
				\end{itemize}%
			\fi%
			\beamer@atendnote%
		}%
		\gdef\beamer@notesactions{}%
	}
}

\makeatother
%=================================Конец=====================================



%===========================   Выравнивание по ширине в block  ===========
\addtobeamertemplate{block begin}{}{\justifying}
\renewcommand{\raggedright}{\leftskip=0pt \rightskip=0pt plus 0cm} %global

%============================== Библиография занимает меньше места ======
\setbeamertemplate{bibliography entry title}{}
\setbeamertemplate{bibliography entry location}{}
\setbeamertemplate{bibliography entry note}{}
\setbeamertemplate{bibliography item}{\insertbiblabel}
%===========================================================================

%===========Заменяем библиографию с квадратных скобок на точку
\makeatletter
\renewcommand{\@biblabel}[1]{#1.}
\makeatother
%============================================================

%============ Черный шрифт для библиографии  ==================================
\setbeamercolor{bibliography item}{fg=black}
\setbeamercolor*{bibliography entry title}{fg=black}
\setbeamercolor*{bibliography entry author}{fg=black}
\setbeamercolor*{bibliography entry journal}{fg=black}
\setbeamercolor*{bibliography entry location}{fg=black}
%==============================================================================


%================== Общие макросы и команды ==========================

% просто чтобы не писать много буковов
\newcommand{\tb}[1]{\textsubscript{#1}}
\newcommand{\tp}[1]{\textsuperscript{#1}}

%============ The fullFrameMovie command defenition ==========================
% from early pdfpc sty 
% Package: textpos is required for textblock*
\usepackage[absolute,overlay]{textpos}


% fullFrameMovie
%
% Arguments:
%
%   [optional]: movie-options, seperated by &
%       Supported options: loop, start=N, end=N, autostart
%   Default: autostart&loop
%
%   1. Movie file
%   2. Poster image
%   3. Any text on the slide, or nothing (e.g. {})
%
% Example:
%   \fullFrameMovie[loop&autostart]{apollo17.avi}{apollo17.jpg}{\copyrightText{Apollo 17, NASA}}
%
\newcommand{\fullFrameMovie}[4][autostart&loop]
{
	{
			\setbeamercolor{background canvas}{bg=black}


			% to make this work for both horizontally filled and vertically filled images, we create an absolutely
			% positioned textblock* that we force to be the width of the slide.
			% we then place it at (0,0), and then create a box inside of it to ensure that it's always 95% of the vertical
			% height of the frame.  Once we have created an absolutely positioned and sized box, it doesn't matter what
			% goes inside -- it will always be vertically and horizontally centered
			\frame[plain]
			{
				\begin{textblock*}{\paperwidth}(0\paperwidth,0\paperheight)
					\centering
					\vbox to 0.99\paperheight {
						\vfil{
							\href{run:#2?autostart&#1}{\includegraphics[width=\paperwidth,height=0.95\paperheight,keepaspectratio]{#3}}
						}
						\vfil
					}
				\end{textblock*}
				#4
			}
		}
}

\newcommand{\fullFrameMultimedia}[3]
{
	{
			\setbeamercolor{background canvas}{bg=black}
			\usebackgroundtemplate[default]


			% to make this work for both horizontally filled and vertically filled images, we create an absolutely
			% positioned textblock* that we force to be the width of the slide.
			% we then place it at (0,0), and then create a box inside of it to ensure that it's always 95% of the vertical
			% height of the frame.  Once we have created an absolutely positioned and sized box, it doesn't matter what
			% goes inside -- it will always be vertically and horizontally centered
			\frame[plain]
			{
				\begin{textblock*}{\paperwidth}(0\paperwidth,0\paperheight)
					\centering
					\vbox to 0.99\paperheight {
						\vfil{
							% \href{run:#2?autostart&#1}{\includegraphics[width=\paperwidth,height=0.95\paperheight,keepaspectratio]{#3}}
							\movie[width=\paperwidth,height=0.95\paperheight]{\includegraphics[width=\paperwidth,keepaspectratio]{#2}}{#1}

						}
						\vfil
					}
				\end{textblock*}
				#3
			}
		}
}
% inlineMovie
%
% Arguments:
%
%   [optional]: movie-options, seperated by &
%       Supported options: loop, start=N, end=N, autostart
%   Default: autostart&loop
%
%   1. Movie file
%   2. Poster image
%   3. size command, such as width=\textwidth
%
% Example:
%   \inlineMovie[loop&autostart&start=5&stop=12]{apollo17.avi}{apollo17.jpg}{height=0.7\textheight}
%
\newcommand{\inlineMovie}[4][autostart&loop]
{
	\href{run:#2?#1}{\includegraphics[#4]{#3}}
}


% copyrightText
%
% Produces small text on the right side of the screen, useful for
% stating copyright or other small notes in movies or images
%
% Arguments:
%
%   [optional]: text color
%       Default: white
%
%   1. Text to be displayed
%
% Example:
%   \copyrightText{Full frame image of: Apollo 17, NASA}
%
\newcommand\copyrightText[2][white]{%
	\begin{textblock*}{\paperwidth}(0\paperwidth,.97\paperheight)%
		\hfill\textcolor{#1}{\tiny#2}\hspace{20pt}
	\end{textblock*}
}

% fullFrameImageZoomed
%
% Produces a slide that contains a full frame image.  Scales down the image
% to fit if the aspect ratio of the slide does not match the image.
%
% Arguments:
%
%   [optional]: color of text on page
%       Default: white
%
%   1. Path to image file
%   2. Any additional content on the frame
%
% Example:
%   \fullFrameImageZoomed{apollo17.jpg}{\copyrightText{Full frame image of: Apollo 17, NASA}}
%
\newcommand{\fullFrameImage}[3][white]
{
	{
			\setbeamercolor{normal text}{bg=black,fg=#1}


			% to make this work for both horizontally filled and vertically filled images, we create an absolutely
			% positioned textblock* that we force to be the width of the slide.
			% we then place it at (0,0), and then create a box inside of it to ensure that it's always 95% of the vertical
			% height of the frame.  Once we have created an absolutely positioned and sized box, it doesn't matter what
			% goes inside -- it will always be vertically and horizontally centered
			\frame
			{
				\begin{textblock*}{\paperwidth}(0\paperwidth,0\paperheight)
					\centering
					\vbox to 0.95\paperheight {
						\vfil{
							\includegraphics[width=\paperwidth,height=0.95\paperheight,keepaspectratio]{#2}
						}
						\vfil
					}
				\end{textblock*}
				#3
			}
		}
}

% fullFrameImageZoomed
%
% Produces a slide that contains a full frame image.  If the aspect ratio
% of the image does not match the slide, it crops the image.
%
% Arguments:
%
%   [optional]: color of text on page
%       Default: black
%
%   1. Path to image file
%   2. Any additional content on the frame
%
% Example:
%   \fullFrameImageZoomed{apollo17.jpg}{\copyrightText{Full frame image of: Apollo 17, NASA}}
%
\newcommand{\fullFrameImageZoomed}[3][black]
{
	{
			\usebackgroundtemplate{\includegraphics[height=\paperheight]{#2}}
			\setbeamercolor{normal text}{bg=black,fg=#1}
			\frame
			{
				#3
			}
		}
}

% Уменьшает вертикальное расстояние в equation
\makeatletter
\g@addto@macro\normalsize{%
	\setlength\belowdisplayskip{-0pt}
	\setlength\abovedisplayskip{-1pt}
}
\makeatother


% handdrawning decorations
% calc, decorations.pathmorphing, patterns,
%TODO write help
\pgfdeclaredecoration{penciline}{initial}{
	\state{initial}[width=+\pgfdecoratedinputsegmentremainingdistance,
		auto corner on length=1mm,]{
		\pgfpathcurveto%
		{% From
			\pgfqpoint{\pgfdecoratedinputsegmentremainingdistance}
			{\pgfdecorationsegmentamplitude}
		}
		{%  Control 1
			\pgfmathrand
			\pgfpointadd{\pgfqpoint{\pgfdecoratedinputsegmentremainingdistance}{0pt}}
			{\pgfqpoint{-\pgfdecorationsegmentaspect
					\pgfdecoratedinputsegmentremainingdistance}%
				{\pgfmathresult\pgfdecorationsegmentamplitude}
			}
		}
		{%TO 
			\pgfpointadd{\pgfpointdecoratedinputsegmentlast}{\pgfpoint{1pt}{1pt}}
		}
	}
	\state{final}{}
}


% Уменьшает отступы в листинге
\makeatletter
\def\nbhline{%
	\noalign{\ifnum0=`}\fi
		\penalty\@M%
		\futurelet\@let@token\LT@@nobreakhline}
	\def\LT@@nobreakhline{%
		\ifx\@let@token\hline
			\global\let\@gtempa\@gobble%
			\gdef\LT@sep{\penalty\@M\vskip\doublerulesep}% <-- change here
		\else
			\global\let\@gtempa\@empty%
			\gdef\LT@sep{\penalty\@M\vskip-\arrayrulewidth}% <-- change here
		\fi
		\ifnum0=`{\fi}%
	\multispan\LT@cols%
	\unskip\leaders\hrule\@height\arrayrulewidth\hfill\cr
	\noalign{\LT@sep}%
	\multispan\LT@cols%
	\unskip\leaders\hrule\@height\arrayrulewidth\hfill\cr
	\noalign{\penalty\@M}%
	\@gtempa}
\makeatother






\def\preambleloaded{Precompiled preamble loaded.}
 \else
\typeout{\preambleloaded}
\fi


%%Some cool feature here, feel free to add some. 
% =======================================================================
% ========= Большой превью в заметках ===================================
% =======================================================================
\setbeamertemplate{note page}%
{%
	{%
			\scriptsize
			\insertvrule{.5\paperheight}{white}
			\vskip-.5\paperheight%
			\nointerlineskip%
			\vbox{
				\hfill\insertslideintonotes{0.5}
				\hskip-1cm\hskip0pt%
				\vskip-0.5\paperheight%
				\nointerlineskip%
				\begin{pgfpicture}{0cm}{0cm}{0cm}{0cm}
					\begin{pgflowlevelscope}{\pgftransformrotate{90}}
						{\pgftransformshift{\pgfpoint{-2cm}{0.2cm}}%
							\pgftext[base,left]{\footnotesize\the\year-\ifnum\month<10\relax0\fi\the\month-\ifnum\day<10\relax0\fi\the\day}}
					\end{pgflowlevelscope}
				\end{pgfpicture}
			}
			\nointerlineskip%
			\vbox to 0.5\paperheight{\vskip0.5em%
				\hbox{\insertshorttitle[width=3cm]}%
				\begin{minipage}{3cm}
					\insertsection \par%
					\insertsubsection%
				\end{minipage}
				\vskip0.5em
				\textbf{Заголовок:} \par
				\hbox{\insertshortframetitle[width=3cm]}%
				\par
				\textbf{Слайд:} \par
				\insertframenumber.\arabic{slidenumber}{}/ \inserttotalframenumber \par
				\vfil}%
		}%
	\vskip-0.025\paperheight%
	\nointerlineskip%
	\insertnote%
}
%================Белый цвет фона в минислайде=================================
\makeatletter
\renewcommand{\insertslideintonotes}[1]{{%
			\begin{pgfpicture}{0cm}{0cm}{#1\paperwidth}{#1\paperheight}
				\begin{pgflowlevelscope}{\pgftransformscale{#1}}%
					\color{white} % меняем тут
					\pgfpathrectangle{\pgfpointorigin}{\pgfpoint{\paperwidth}{\paperheight}}
					\pgfusepath{fill}
					\color{black} %цвет шрифта
					{\pgftransformshift{\pgfpoint{\beamer@origlmargin}{\footheight}}\pgftext[left,bottom]{\copy\beamer@frameboxcopy}}
				\end{pgflowlevelscope}
			\end{pgfpicture}%
		}}
\makeatother
%=====================Конец==================================================

%============================Подсчет количества слайдов с оверлееми===========
\newcounter{slidenumber}

\defbeamertemplate*{footline}{infolines theme frame plus slide}{
	\setcounter{slidenumber}{\insertpagenumber}%
	\addtocounter{slidenumber}{-\insertframestartpage}%
	\addtocounter{slidenumber}{1}%
	\leavevmode%
	\hbox{%
		\begin{beamercolorbox}[wd=.333333\paperwidth,ht=2.25ex,dp=1ex,center]{author in head/foot}%
			\usebeamerfont{author in head/foot}\insertshortauthor~~(\insertshortinstitute)
		\end{beamercolorbox}%
		\begin{beamercolorbox}[wd=.333333\paperwidth,ht=2.25ex,dp=1ex,center]{title in head/foot}%
			\usebeamerfont{title in head/foot}\insertshorttitle
		\end{beamercolorbox}%
		\begin{beamercolorbox}[wd=.333333\paperwidth,ht=2.25ex,dp=1ex,right]{date in head/foot}%
			\usebeamerfont{date in head/foot}\insertshortdate{}\hspace*{2em}
			\insertframenumber.\arabic{slidenumber}{}/ \inserttotalframenumber\hspace*{2ex}
		\end{beamercolorbox}}%
	\vskip0pt%
}
%require to set below
%\setbeamertemplate{footline}[infolines theme frame plus slide]



%====================Буллет в note[item]======================================
\makeatletter
\def\beamer@setupnote{%
	\gdef\beamer@notesactions{%
		\beamer@outsideframenote{%
			\beamer@atbeginnote%
			\beamer@notes%
			\ifx\beamer@noteitems\@empty\else
				\begin{itemize}\itemsep=0pt\parskip=0pt%
					\beamer@noteitems%
				\end{itemize}%
			\fi%
			\beamer@atendnote%
		}%
		\gdef\beamer@notesactions{}%
	}
}

\makeatother
%=================================Конец=====================================



%===========================   Выравнивание по ширине в block  ===========
\addtobeamertemplate{block begin}{}{\justifying}
\renewcommand{\raggedright}{\leftskip=0pt \rightskip=0pt plus 0cm} %global

%============================== Библиография занимает меньше места ======
\setbeamertemplate{bibliography entry title}{}
\setbeamertemplate{bibliography entry location}{}
\setbeamertemplate{bibliography entry note}{}
\setbeamertemplate{bibliography item}{\insertbiblabel}
%===========================================================================

%===========Заменяем библиографию с квадратных скобок на точку
\makeatletter
\renewcommand{\@biblabel}[1]{#1.}
\makeatother
%============================================================

%============ Черный шрифт для библиографии  ==================================
\setbeamercolor{bibliography item}{fg=black}
\setbeamercolor*{bibliography entry title}{fg=black}
\setbeamercolor*{bibliography entry author}{fg=black}
\setbeamercolor*{bibliography entry journal}{fg=black}
\setbeamercolor*{bibliography entry location}{fg=black}
%==============================================================================


%================== Общие макросы и команды ==========================

% просто чтобы не писать много буковов
\newcommand{\tb}[1]{\textsubscript{#1}}
\newcommand{\tp}[1]{\textsuperscript{#1}}

%============ The fullFrameMovie command defenition ==========================
% from early pdfpc sty 
% Package: textpos is required for textblock*
\usepackage[absolute,overlay]{textpos}


% fullFrameMovie
%
% Arguments:
%
%   [optional]: movie-options, seperated by &
%       Supported options: loop, start=N, end=N, autostart
%   Default: autostart&loop
%
%   1. Movie file
%   2. Poster image
%   3. Any text on the slide, or nothing (e.g. {})
%
% Example:
%   \fullFrameMovie[loop&autostart]{apollo17.avi}{apollo17.jpg}{\copyrightText{Apollo 17, NASA}}
%
\newcommand{\fullFrameMovie}[4][autostart&loop]
{
	{
			\setbeamercolor{background canvas}{bg=black}


			% to make this work for both horizontally filled and vertically filled images, we create an absolutely
			% positioned textblock* that we force to be the width of the slide.
			% we then place it at (0,0), and then create a box inside of it to ensure that it's always 95% of the vertical
			% height of the frame.  Once we have created an absolutely positioned and sized box, it doesn't matter what
			% goes inside -- it will always be vertically and horizontally centered
			\frame[plain]
			{
				\begin{textblock*}{\paperwidth}(0\paperwidth,0\paperheight)
					\centering
					\vbox to 0.99\paperheight {
						\vfil{
							\href{run:#2?autostart&#1}{\includegraphics[width=\paperwidth,height=0.95\paperheight,keepaspectratio]{#3}}
						}
						\vfil
					}
				\end{textblock*}
				#4
			}
		}
}

\newcommand{\fullFrameMultimedia}[3]
{
	{
			\setbeamercolor{background canvas}{bg=black}
			\usebackgroundtemplate[default]


			% to make this work for both horizontally filled and vertically filled images, we create an absolutely
			% positioned textblock* that we force to be the width of the slide.
			% we then place it at (0,0), and then create a box inside of it to ensure that it's always 95% of the vertical
			% height of the frame.  Once we have created an absolutely positioned and sized box, it doesn't matter what
			% goes inside -- it will always be vertically and horizontally centered
			\frame[plain]
			{
				\begin{textblock*}{\paperwidth}(0\paperwidth,0\paperheight)
					\centering
					\vbox to 0.99\paperheight {
						\vfil{
							% \href{run:#2?autostart&#1}{\includegraphics[width=\paperwidth,height=0.95\paperheight,keepaspectratio]{#3}}
							\movie[width=\paperwidth,height=0.95\paperheight]{\includegraphics[width=\paperwidth,keepaspectratio]{#2}}{#1}

						}
						\vfil
					}
				\end{textblock*}
				#3
			}
		}
}
% inlineMovie
%
% Arguments:
%
%   [optional]: movie-options, seperated by &
%       Supported options: loop, start=N, end=N, autostart
%   Default: autostart&loop
%
%   1. Movie file
%   2. Poster image
%   3. size command, such as width=\textwidth
%
% Example:
%   \inlineMovie[loop&autostart&start=5&stop=12]{apollo17.avi}{apollo17.jpg}{height=0.7\textheight}
%
\newcommand{\inlineMovie}[4][autostart&loop]
{
	\href{run:#2?#1}{\includegraphics[#4]{#3}}
}


% copyrightText
%
% Produces small text on the right side of the screen, useful for
% stating copyright or other small notes in movies or images
%
% Arguments:
%
%   [optional]: text color
%       Default: white
%
%   1. Text to be displayed
%
% Example:
%   \copyrightText{Full frame image of: Apollo 17, NASA}
%
\newcommand\copyrightText[2][white]{%
	\begin{textblock*}{\paperwidth}(0\paperwidth,.97\paperheight)%
		\hfill\textcolor{#1}{\tiny#2}\hspace{20pt}
	\end{textblock*}
}

% fullFrameImageZoomed
%
% Produces a slide that contains a full frame image.  Scales down the image
% to fit if the aspect ratio of the slide does not match the image.
%
% Arguments:
%
%   [optional]: color of text on page
%       Default: white
%
%   1. Path to image file
%   2. Any additional content on the frame
%
% Example:
%   \fullFrameImageZoomed{apollo17.jpg}{\copyrightText{Full frame image of: Apollo 17, NASA}}
%
\newcommand{\fullFrameImage}[3][white]
{
	{
			\setbeamercolor{normal text}{bg=black,fg=#1}


			% to make this work for both horizontally filled and vertically filled images, we create an absolutely
			% positioned textblock* that we force to be the width of the slide.
			% we then place it at (0,0), and then create a box inside of it to ensure that it's always 95% of the vertical
			% height of the frame.  Once we have created an absolutely positioned and sized box, it doesn't matter what
			% goes inside -- it will always be vertically and horizontally centered
			\frame
			{
				\begin{textblock*}{\paperwidth}(0\paperwidth,0\paperheight)
					\centering
					\vbox to 0.95\paperheight {
						\vfil{
							\includegraphics[width=\paperwidth,height=0.95\paperheight,keepaspectratio]{#2}
						}
						\vfil
					}
				\end{textblock*}
				#3
			}
		}
}

% fullFrameImageZoomed
%
% Produces a slide that contains a full frame image.  If the aspect ratio
% of the image does not match the slide, it crops the image.
%
% Arguments:
%
%   [optional]: color of text on page
%       Default: black
%
%   1. Path to image file
%   2. Any additional content on the frame
%
% Example:
%   \fullFrameImageZoomed{apollo17.jpg}{\copyrightText{Full frame image of: Apollo 17, NASA}}
%
\newcommand{\fullFrameImageZoomed}[3][black]
{
	{
			\usebackgroundtemplate{\includegraphics[height=\paperheight]{#2}}
			\setbeamercolor{normal text}{bg=black,fg=#1}
			\frame
			{
				#3
			}
		}
}

% Уменьшает вертикальное расстояние в equation
\makeatletter
\g@addto@macro\normalsize{%
	\setlength\belowdisplayskip{-0pt}
	\setlength\abovedisplayskip{-1pt}
}
\makeatother


% handdrawning decorations
% calc, decorations.pathmorphing, patterns,
%TODO write help
\pgfdeclaredecoration{penciline}{initial}{
	\state{initial}[width=+\pgfdecoratedinputsegmentremainingdistance,
		auto corner on length=1mm,]{
		\pgfpathcurveto%
		{% From
			\pgfqpoint{\pgfdecoratedinputsegmentremainingdistance}
			{\pgfdecorationsegmentamplitude}
		}
		{%  Control 1
			\pgfmathrand
			\pgfpointadd{\pgfqpoint{\pgfdecoratedinputsegmentremainingdistance}{0pt}}
			{\pgfqpoint{-\pgfdecorationsegmentaspect
					\pgfdecoratedinputsegmentremainingdistance}%
				{\pgfmathresult\pgfdecorationsegmentamplitude}
			}
		}
		{%TO 
			\pgfpointadd{\pgfpointdecoratedinputsegmentlast}{\pgfpoint{1pt}{1pt}}
		}
	}
	\state{final}{}
}


% Уменьшает отступы в листинге
\makeatletter
\def\nbhline{%
	\noalign{\ifnum0=`}\fi
		\penalty\@M%
		\futurelet\@let@token\LT@@nobreakhline}
	\def\LT@@nobreakhline{%
		\ifx\@let@token\hline
			\global\let\@gtempa\@gobble%
			\gdef\LT@sep{\penalty\@M\vskip\doublerulesep}% <-- change here
		\else
			\global\let\@gtempa\@empty%
			\gdef\LT@sep{\penalty\@M\vskip-\arrayrulewidth}% <-- change here
		\fi
		\ifnum0=`{\fi}%
	\multispan\LT@cols%
	\unskip\leaders\hrule\@height\arrayrulewidth\hfill\cr
	\noalign{\LT@sep}%
	\multispan\LT@cols%
	\unskip\leaders\hrule\@height\arrayrulewidth\hfill\cr
	\noalign{\penalty\@M}%
	\@gtempa}
\makeatother




\usetheme[block=fill,progressbar=frametitle,background=light]{moloch}
% Цвета ИТМО
\usecolortheme{moloch-itmo}

% Тут включается отображение заметок.
\setbeameroption{show notes on second screen=top} % расположение заметок 

\setbeamercovered{transparent} % полупрозрачность для \uncover

\hypersetup{
	colorlinks=true,
	linkcolor=black, % дефолтный цвет из темы
	filecolor=magenta,
	urlcolor=cyan,
	pdftitle={Презентация ВКР},
	pdfpagemode=FullScreen,
}
%=====================================ЛОГО======================================

% Логотип на Титульной странице
% Тут логотип отправлен вправо коммандой \hfill
% \titlegraphic{\hfill\includegraphics[width=0.3\textwidth]{./inc/LogoSPbFTU-2.pdf}}
\titlegraphic{\includegraphics[width=0.2\textwidth]{itmo/logo_basic_russian_white.pdf}}
% Логотип на остальных страницах
% \logo{
% 	\begin{minipage}{5em}
% 		\begin{block}{}
% 			\centering
% 			\bfseries
% 			Powered by \\ {\small\textrm\LaTeX{}}  \\  and LAB\,60
% 		\end{block}
% 	\end{minipage}\hspace{1em}
% }


% Фоновый рисунок обычный
% \setbeamertemplate{background canvas}{\includegraphics[width=\paperwidth]{background}}
% Фоновый рисунок тайловый
% \setbeamertemplate{background canvas}{%
% 	\vbox{%
% 		\foreach\y in {1,...,2}{%
% 				\foreach\x in {1,...,2}{%
% 						\includegraphics[width=.5\paperwidth]{background}%
% 					}%
% 				\linebreak
% 			}%
% 	}%
% }
% Настройка слайда с названием секции
\AtBeginSection{
	\begingroup
	\setbeamertemplate{background canvas}[default]
	% \setbeamercolor{background canvas}{bg=grayFTU} % используется цвет из темы FTU
	\logo{}
	\frame{\sectionpage}
	\endgroup
}

% Использовать блоки с тенью
\setbeamertemplate{blocks}[rounded][shadow=true]
% Визуальный эффект смены слайдов
\addtobeamertemplate{background canvas}{\transfade[duration=2]}{}

%======================== Изменение ЦВЕТА темы ======================================
%\setbeamercolor{alerted text}{fg=orange}
%\setbeamercolor{background canvas}{bg=white}
%================цвет блока для всего документа==============
%\setbeamercolor{block body alerted}{bg=normal text.bg!90!black}
%\setbeamercolor{block body}{bg=normal text.bg!90!black}
%\setbeamercolor{block body example}{bg=normal text.bg!90!black}
%\setbeamercolor{block title alerted}{use={normal text,alerted text},fg=alerted text.fg!75!normal text.fg,bg=normal text.bg!75!black}
%\setbeamercolor{block title}{bg=blue}
%\setbeamercolor{block title example}{use={normal text,example text},fg=example text.fg!75!normal text.fg,bg=normal text.bg!75!black}
%\setbeamercolor{fine separation line}{}
%\setbeamercolor{frametitle}{fg=brown}
%\setbeamercolor{item projected}{fg=black}
%\setbeamercolor{normal text}{bg=black,fg=yellow}
%\setbeamercolor{palette sidebar primary}{use=normal text,fg=normal text.fg}
%\setbeamercolor{palette sidebar quaternary}{use=structure,fg=structure.fg}
%\setbeamercolor{palette sidebar secondary}{use=structure,fg=structure.fg}
%\setbeamercolor{palette sidebar tertiary}{use=normal text,fg=normal text.fg}
%\setbeamercolor{section in sidebar}{fg=brown}
%\setbeamercolor{section in sidebar shaded}{fg=grey}
%\setbeamercolor{separation line}{}
%\setbeamercolor{sidebar}{bg=red}
%\setbeamercolor{sidebar}{parent=palette primary}
%\setbeamercolor{structure}{bg=black, fg=green}
%\setbeamercolor{subsection in sidebar}{fg=brown}
%\setbeamercolor{subsection in sidebar shaded}{fg=grey}
%\setbeamercolor{title}{fg=brown}
%\setbeamercolor{titlelike}{fg=brown}
%===============Конец описания цветов=======================


%=============================БЛОКИ==========================================
%цвет блока для всего документа
%\setbeamercolor{block title}{fg=orange!20!black,bg=green}%bg=background, fg= foreground
%\setbeamercolor{block body}{bg=green!10,fg=black}%bg=background, fg= foreground

%создаем специальный блок другого цвета theorem, exampleblock, alertblock уже есть!
\newenvironment<>{problock}[1]{%
	\begin{actionenv}#2%
		\def\insertblocktitle{#1}%
		\par%
		\mode<presentation>{%
			\setbeamercolor{block title}{fg=white,bg=orange!20!black}
			\setbeamercolor{block body}{fg=black,bg=yellow!20}
			\setbeamercolor{itemize item}{fg=orange!20!black}
			\setbeamertemplate{itemize item}[triangle]
		}%
		\usebeamertemplate{block begin}}
		{\par\usebeamertemplate{block end}\end{actionenv}}

%============TIKZ Styles=========================
%
% \tikzstyle{subst} = [rounded rectangle,
% thick,
% inner sep=5pt,
% minimum size=1cm,
% draw=red!50!black!50,
% top color=white,
% bottom color=red!50!black!50,
% font=\itshape,
% drop shadow]
% \tikzstyle{process} = [ellipse,
% thick,
% minimum size=1cm,
% draw=blue!50!black!50,
% top color=white,
% bottom color=blue!50!black!20,
% drop shadow]
%============END of TIKZ Styles=====================

% Настройка tcolorbox
\tcbuselibrary{breakable}
\tcbuselibrary{skins}

\newtcolorbox{mybox}[2][]%
{%
	attach boxed title to top center
		= {yshift=-8pt},
	%colback      = blue!5!white,
	%colframe     = blue!75!black,
	fonttitle    = \bfseries,
	%opacityback=0.8,
	%opacitybacktitle=0.75,
	%colbacktitle = blue!85!black,
	breakable,
	left=0mm,
	right=0mm,
	%left skip=-2mm,
	%right skip=-2mm,
	title        = #2,#1,
	skin=enhanced
}
\newtcolorbox{process}[2][]%
{%
	attach boxed title to top center
		= {yshift=-8pt},
	%colback      = blue!5!white,
	%colframe     = blue!75!black,
	fonttitle    = \bfseries,
	%opacityback=0.8,
	%opacitybacktitle=0.75,
	colbacktitle = orange,
	breakable,
	left=0mm,
	right=0mm,
	%left skip=-2mm,
	%right skip=-2mm,
	title        = #2,#1,
	skin=enhanced,
}

\newtcolorbox{autotherm}[2][]%
{%
	attach boxed title to top center
		= {yshift=-8pt},
	%colback      = blue!5!white,
	%colframe     = blue!75!black,
	fonttitle    = white,
	%opacityback=0.8,
	%opacitybacktitle=0.75,
	colupper=white,
	colback = orange,
	breakable,
	left=0mm,
	right=0mm,
	%left skip=-2mm,
	%right skip=-2mm,
	title        = #2,#1,
	skin=enhanced,
}
\newtcolorbox{timeline}[2][]%
{%
	attach boxed title to top center
		= {yshift=-8pt},
	%colback      = blue!5!white,
	%colframe     = blue!75!black,
	%fonttitle    = white,
	%opacityback=0.8,
	%opacitybacktitle=0.75,
	%colupper=white,
	%colback = white,
	colbacktitle = white,
	breakable,
	left=0mm,
	right=0mm,
	%left skip=-2mm,
	%right skip=-2mm,
	title        = #2,#1,
	skin=enhanced,
}
\newtcbox{\xmybox}{
	on line,
	arc=7pt,
	before upper={\rule[-3pt]{0pt}{5pt}},
	boxrule=1pt,
	boxsep=0pt,
	left=2pt,
	right=2pt,
	top=3pt,
	bottom=2pt
}

% Новый тип колонки для примера
% \newcolumntype{Y}{>{\raggedleft\arraybackslash}X}

\tcbset{tab1/.style={fonttitle=\bfseries\large,fontupper=\normalsize\sffamily,
			colback=green!10!white,colframe=green!75!black,colbacktitle=green!40!white,
			coltitle=black,center title,freelance,frame code={
					\foreach \n in {north east,north west,south east,south west}
						{\path [fill=darkgreen] (interior.\n) circle (3mm); };},}}

\tcbset{tab2/.style={enhanced,fonttitle=\bfseries,fontupper=\normalsize\sffamily,
			colback=green!10!white,colframe=green!50!black,colbacktitle=green!40!white,
			coltitle=black,center title}}


%TODO
% \useinnertheme{tcolorbox}

% Настройка листинга для кода

\lstset{ %
	language={[latex]TeX},          % язык программирования
	inputencoding=utf8,             % кодировка
	autogobble=true,                % убирает пробелы в начале строки
	basicstyle=\ttfamily\footnotesize,       % the size of the fonts that are used for the code
	numbers=left,                   % where to put the line-numbers
	numberstyle=\tiny\color{gray},  % the style that is used for the line-numbers
	stepnumber=1,                   % the step between two line-numbers. If it's 1, each line
	% will be numbered
	numbersep=0pt,                  % how far the line-numbers are from the code
	% backgroundcolor=\color{white},  % choose the background color. You must add \usepackage{color}
	showspaces=false,               % show spaces adding particular underscores
	showstringspaces=false,         % underline spaces within strings
	showtabs=false,                 % show tabs within strings adding particular underscores
	% frame=single,                   % adds a frame around the code
	rulecolor=\color{black},        % if not set, the frame-color may be changed on line-breaks within not-black text (e.g. commens (green here))
	tabsize=2,                      % sets default tabsize to 2 spaces
	captionpos=b,                   % sets the caption-position to bottom
	title=\lstname,                 % show the filename of files included with \lstinputlisting;
	% also try caption instead of title
	keywordstyle=\color{green},       % keyword style
	commentstyle=\color{blue},      % comment style
	stringstyle=\color{gray},      % string literal style
	escapeinside={\%*}{*)},         % if you want to add a comment within your code
	extendedchars=\true,            % поддержка расширенных символов
	keepspaces = true               %!!!! пробелы в комментариях
	texcl=true,
	breaklines=true,                % разрывать строку
	breakatwhitespace=true,         % разрыв по пробелам
	aboveskip=-0.25 \baselineskip,   % верхний интервал
	belowskip=-1.5 \baselineskip,   % нижный интервал
	morekeywords={*,note,... }      % if you want to add more keywords to the set
}
\renewcommand\lstlistingname{Листинг}


\useinnertheme[
	rounded,
	% shadow
]{tcolorbox}

% \tcbsetforeverylayer{
% 	borderline={1pt}{0pt}{greenFTU!40!black}
% }





%============================================================================
%=========================        ===========================================
%=========================        ===========================================
%=========================Заглавие===========================================
%=========================        ===========================================
%============================================================================
\title[КММ]{Компьютерные методы и моделирование}

\subtitle[Лекция 8]{Статистическое планирование эксперимента}

\date{\today}

%======================== Слайды ============================================
\begin{document}
\setbeamertemplate{itemize items}[square] % bullet style in itemize env
%\setbeamertemplate{enumerate items}[square] %circle,square
\setbeamertemplate{itemize items}[circle]

{
	% \setbeamertemplate{background canvas}[default]
	\setbeamertemplate{background canvas}{%
		\includegraphics[width=\paperwidth,height=\paperheight]{itmo/vitmo_snakes.jpg}%
	}%
	\begin{frame}
		\titlepage
	\end{frame}
}
\note{
	Основными недостаткам и математических моделей, полученных с помощью классического регрессионного анализа, являются корреляция между коэффициентами и трудности в оценке ошибки расчетного значения параметра оптимизации.
	Недостатки классического регрессионного анализа затрудняют его применение, так как какая-либо физико-химическая интерпретация уравнений регрессии, их переменных и эффектов их взаимодействия затруднительна.
	Наряду с этим регрессионный анализ является весьма эффективным с точки зрения математической статистики и удобным для экспериментатора методом, позволяющим представить в компактной форме всю информацию о процессе, полученную из экспериментов
	В основе методов статистического планирования экспериментов лежит использование упорядоченного плана расположения точек в факторном пространстве и переход к новой системе координат.

}

\begin{frame}[t]{Планы первого порядка}

	\begin{columns}
		\column{0.45\textwidth}
		\begin{block}{}
			\center{\includegraphics[width=1\linewidth]{./images/statG1.png}}
		\end{block}
		\column{0.45\textwidth}
		\begin{block}{}
			\center{\includegraphics[width=1\linewidth]{./images/statG2.png}}
		\end{block}
	\end{columns}
	\begin{block}{}
		\begin{equation}
			x_{ij} = \dfrac{2z_{ij} - z_{1j} - z_{2i}}{z_{2j} - z_{1j}}
		\end{equation}
	\end{block}

	\note<+>{
		Построение плана первого порядка начинается с выбора интервалов изменения факторов.
		Обозначим через $z_{1j}$ и $z_{2j}$ нижнюю и верхнюю границы изменения факторов;
		через $y$ --- параметр оптимизации.
		Тогда, например, для двухфакторной задачи область факторного пространства, подлежащая изучению, будет иметь вид прямоугольника
		(рис. 4.1) с координатами угловых точек: 1(z21, z12), 2(z11 Z12), 3(z21, Z22), 4(z11 z22).
		Координаты центра изучаемой области обозначим через Z0j, а координаты любой точки --- через Zij.

	}
	\note<+>{
		Закодируем значения переменных по формуле
		$$
			x_{ij} = \dfrac{2z_{ij} - z_{1j} - z_{2i}}{z_{2j} - z_{1j}}
		$$
		Координаты точек 1; 2; 3; 4 записывают в виде таблицы, называемой матрицей планирования эксперимента.

	}


\end{frame}

\begin{frame}[t]{Матрица планирования эксперимента}
	\begin{block}{}
		\begin{center}
			\begin{tabular}[c]{c|c|c|c}
				\hline
				№ точки & $x_1$ & $x_2$ & $y$   \\
				\hline
				1       & $+1$  & $-1$  & $y_1$ \\
				2       & $-1$  & $-1$  & $y_2$ \\
				3       & $+1$  & $+1$  & $y_3$ \\
				4       & $-1$  & $+1$  & $y_4$ \\
				\hline
			\end{tabular}
		\end{center}

	\end{block}
	\begin{block}{}
		Построенный таким способом план экспериментов обладает рядом весьма цепных свойств:
		\begin{itemize}
			\item содержит все комбинации двух значений переменной, равных (+1) и (-1).
			\item $\sum_{i=1}^m x_{ij}^2 = m $
			\item $\sum_{i=1}^m x_{ij} = 0 $
			\item $\sum_{i=1}^m x_{ij} x_{iv} = 0 \quad j \neq v $
		\end{itemize}
	\end{block}
	\note<+>{%
		Матрица планирования вместе с результатами экспериментов имеет вид, представленный в табл.
		Построенный таким способом план экспериментов обладает рядом весьма цепных свойств:
		\begin{itemize}
			\item содержит все комбинации двух значений переменной, равных (+1) и (-1).
			\item $\sum_{i=1}^m x_{ij}^2 = m $
			\item $\sum_{i=1}^m x_{ij} = 0 $
			\item $\sum_{i=1}^m x_{ij} x_{iv} = 0 \quad j \neq v $
		\end{itemize}
		Свойство 4 носит название «ортогональности».
		Отсюда такие планы называют ортогональными.
		Их называют также планами полного факторного эксперимента первого порядка.
		Второе название вытекает из первого свойства.

	}
	\note<+>{%
		Свойство ортогональности дает возможность избавиться от недостатков классического регрессионпого анализа и значительно снизить вычислительные трудности, возникающие при расчете коэффициентов регрессии.
		Кроме того, переход к безразмерным переменным по формуле (4.1) делает все факторы равноправными внутри изучаемой области, т. е. дает возможность на основании величин и знаков коэффициентов судить об их роли в процессе.

	}


\end{frame}

\begin{frame}[t]{Пример двухфакторного эксперимента}

	\begin{block}{}
		\begin{equation}
			f^v = 2^2 = 4,
		\end{equation}
		\center{\includegraphics[width=1\linewidth]{./images/statT1.png}}
	\end{block}
	\note<+>{%
		В качестве примера рассмотрим попкорн. Мы будем стараться оптимизировать
		количество лопнувших зерен.
		В самом эксперименте у нас будут 2 исследуемых фактора, которые принимают по 2
		значения: А - время нагрева (160 и 200 с) и В - тип попкорна (желтый и белый).
		Можно легко посчитать, что число экспериментов будет 4.
		В общем случае для расчета количества экспериментов
		используют следующую формулу:
		$$
			f^v = 2^2 = 4,
		$$
		где f - число факторов, а v - число значений, принимаемых фактором.

	}
	\note<+>{%
		Составим таблицу эксперимента (табл. 3). Обозначим низкое и высокое значение фактора как - и + соответственно (для категориального - не важно, выбираем любой).
		Тогда для А: - = 160, + = 200, для В: - = белый, + = желтый.

		Для получения информативных результатов важно:
		не использовать экстремальных значений для факторов (иначе на них оказывается множество влияний и они слишком сильно будут отличаться друг от друга, что увеличит погрешности);
		всегда проводить эксперименты в случайном порядке!
		Только так мы сможем избавиться от систематической погрешности и возникающих дополнительных связях между величинами.

	}
\end{frame}

\begin{frame}[t]{Обарботка}
	\begin{block}{}
		\center{\includegraphics[ width=0.7\linewidth]{./images/statG3.png}}

		Кубическая диаграмма 2- ух факторного эксперимента с попкорном. На диаграмме изображены изолинии.
	\end{block}

	\note<+>{%
		Итак, результаты получены.
		Само время проводить анализ.
		Начинать всегда лучше всего с визуализации (так уж устроено наше мышление).
		Визуализация факторного эксперимента называется кубической диаграммой (графиком/планом, cube plot).
		Она приведен на рис. 3.
		Эта диаграмма показывает эффект от каждого фактора в соответствующем углу
		квадрата или куба (2 или 3 факторные эксперименты).
	}
	\note<+>{%
		Начнем с оценки эффекта от времени.
		При увеличении времени готовки для желтого попкорна, результат увеличивается с 62 до 80 лопнувших зерен (ЛЗ).
		Мы видим рост на 18 единиц.
		Для белого попкорна мы видим изменение с 52 до 74 ЛЗ, то есть рост на 22 единицы.
		Итак, в среднем мы видим увеличение на 20 единиц при увеличении продолжительности нагрева со 160 до 200 секунд.

		Далее давайте оценим разницу между двумя типами попкорна.
		Зафиксируем время нагрева и посмотрим на эффект от перехода от белого к желтому попкорну: с 74 до 80 для 200 с и с 52 до 62 для 160 с.
		В среднем мы видим увеличение на 8 единиц при переходе от белого к желтому попкорну.
		Убедитесь, что ваша интерпретация соответствует кубической диаграмме. Эта визуализация очень важна для самопроверки результатов анализа

	}
	\note<+>{%
		Но помимо результатов, на кубической диаграмме отображены еще и контурные
		линии (contour plot, их еще называют изолиниями, isolines). Они обозначают
		область, в которой значение измеряемого признака остается постоянным (на 1
		линии количество лопнувших зерен будет постоянным). Их рисуют начиная с
		любого угла кубической диаграммы, значение в котором не является
		максимальным или минимальным. Затем ищут это же значение на
		противоположенной стороне квадрата и проводят линию в соответствии с
		предполагаемым уровнем результата. Для проверки кривизны линии нужно
		рассчитать наше фиксированное значение для середины шкалы.

	}
	\note<+>{%
		Затем рисуем вторую линию аналогично для значения в 74.
		Остальные рисуем параллельно полученным линиям.
		Благодаря изолиниям можно быстро понять, куда начинать движение для оптимизации результата, т.е. по направлению к нашей цели.
		Например, если цель — максимизировать количество лопнувших зерен, то двигаться нужно перпендикулярно изолиниям в верхний правый угол.
		В данном случае это означает, что мы должны взять желтый попкорн и увеличить время приготовления (что вполне интуитивно понятно из кубической диаграммы).

	}
	\note<+>{%
		Такой подход к оптимизации (с использованием изолиний) помогает нам
		определиться с метом проведения следующего эксперимента.
		Контурная диаграмма - это наш градиент (gradient, путь, по которому пройдут наши эксперименты для подтверждения/опровержения закона или теории).

	}
\end{frame}

\begin{frame}[t]{Диаграмма взаимодействия}
	\begin{block}{}
		\center{\includegraphics[width=0.6\linewidth]{./images/statG4.png}}

		Диаграмма взаимодействия для 2 факторного эксперимента с попкорном.

	\end{block}

	\note<+>{%
		Отмечу, что есть еще один способ визуализации - диаграмма взаимодействия (interaction plot) (рис. 4).

		Обратите внимание, что эти две линии практически параллельны, что означает, что в исследуемой системе практически отсутствует взаимодействие.
		Выбор переменной для диаграммы взаимодействия не играет большой роли и мы могли бы выбрать другую переменную для обозначения на горизонтальной оси.
		Для всех описанных методов визуализации не требуется какое-либо программное обеспечение.

	}
	\note<+>{%
		Вы можете использовать эти методы визуализации как для числовых, так и для категориальных факторов.
		Все это демонстрирует явное преимущество такого подхода к эксперименту: мы можем быстро интерпретировать результаты, используя простые графические инструменты, элементарную математику и лист бумаги.

		Тот факт, что все так просто, означает, что результатами будет легко поделиться с менеджерами или коллегами на работе.

	}
\end{frame}

\begin{frame}[t]{Уравнение}
	\begin{block}{}
		\begin{equation}
			\hat{y} = a0 + a1 \cdot x_A + a2 \cdot x_B
		\end{equation}
		\begin{itemize}
			\item a0 --- базовый результат (intercept), который мы ожидаем увидеть при отсутствии влияния (когда закодированные значения факторов = 0).
			\item a1 - коэффициент влияния фактора А (его закодированного значения).
			      \begin{equation}
				      a1 = \dfrac{(80-62)+(74-52)}{2} \cdot \dfrac{1}{2}
			      \end{equation}
			\item a2 - коэффициент влияния фактора В, зависящий от типа зерен.
		\end{itemize}

		Учитывая приведенное описание наша модель будет:
		\begin{equation}
			\hat{y} = 67 + 10 \cdot x_A + 4 \cdot x_B
		\end{equation}
	\end{block}
	\note<+>{%
		Мы рассмотрели пример планирования, проведения и анализа эксперимента.
		Но что это нам дает? Как мы можем представить и использовать полученные данные? Ответ - построить прогноз (модель, уравнение регрессии).
		В рамках нашего курса мы будем рассматривать только линейные модели (за небольшим исключением).
		Такие модели наиболее универсальные (любую гладкую и монотонную функцию можно представить как набор линейных отрезков).

	}
	\note<+>{%
		В случае нашего "попкорн-эксперимента" (2-ух факторный эксперимент),
		полученная модель состоит из 3 частей:
		$$
			\hat{y} = a0 + a1 \cdot x_A + a2 \cdot x_B
		$$
		где,
		1. a0 - базовый результат (intercept), который мы ожидаем увидеть при
		отсутствии влияния (когда закодированные значения факторов = 0).
		Этот коэффициент рассчитывается как среднее из 4 значений на кубической
		диаграмме (т.е. ее центр).

	}
	\note<+>{%

		2. a1 - коэффициент влияния фактора А (его закодированного значения), зависит от времени приготовления.
		Рассчитывается как средняя нормированная разница между высоким и низким значением фактора:
		$$
			a1 = \dfrac{\dfrac{(80-62)+(74-52)}{2} }{2}
		$$

		Обратите внимание, нормировка подразумевает расчет
		коэффициента для единичного изменения фактора (т.е. с -1 до 0 или от 0 до
		+1), поэтому мы должны разделить усредненное значение на 2.

		3. a2 - коэффициент влияния фактора В, зависящий от типа зерен.
		Рассчитывается аналогично пункту 2.

	}
	\note<+>{%
		Учитывая приведенное описание наша модель будет:
		$$
			\hat{y} = 67 + 10 \cdot x_A + 4 \cdot x_B
		$$
	}

\end{frame}

\begin{frame}[t]{Взаимодействие факторов}
	\begin{block}{}
		\center{\includegraphics[width=0.7\linewidth]{./images/statG5.png}}

	\end{block}
	\note<+>{%
		До сих пор мы рассматривали весьма идеальные случаи где нет взаимного влияния факторов друг на друга и на целевую переменную.
		Однако зачастую это не так.

		Пример. Мы пытаемся отмыть руки и проводим 2-ух факторный эксперимент: есть/нет мыла и теплая/холодная вода.
		Можно заметить, что эффект теплой воды усилится при использовании мыла.
		И наоборот, эффект мыла усилится при использовании теплой воды.
		То есть "взаимодействие" говорит о том, что эффект одного фактора зависит от уровня другого фактора.
		Кроме этого, эти взаимодействия обычно симметричны (не не всегда!).
		Т.е. нет разницы будем ли мыть руки в теплой воде с мылом или с мылом в теплой воде, результат будет одинаков.

	}
	\note<+>{%
		Первым показателем наличия взаимосвязи является несимметричность линий на диаграмме взаимодействия или изогнутость изолинии на кубической диаграмме.
		Если вы наблюдаете такие эффекты, то это проявляется двухфакторное взаимодействие (когда поведение одной переменной сильно отличается в зависимости от уровня другой переменной).

		Рассмотрим эксперимент на рис. 5 и рассчитаем все коэффициенты. Эксперимент заключался в анализе влияния времени выпечки (фактор А) и типа подсластителя (фактор В) на вкус печенья (по шкале от 1 до 10).

	}
	\note<+>{%
		Обратите внимание, что изолинии уже не параллельные и изображать их нужно в изогнутом виде (еще раз напомню, на изолинии должно быть одинаковое значение результата).
		Для этого рекомендую провести вспомогательную линию по центру кубической диаграммы.
		Выраженная не параллельность линий сигнализирует о наличии взаимного влияние факторов друг на друга.
		Строго говоря, при анализе эксперимента нужно всегда строить модель с учетом взаимного влияния и исключать его только если коэффициент перед этим фактором в модели очень мал.
		Рассчитаем получившуюся модель.

	}

\end{frame}

\begin{frame}[t]{Взаимодействие факторов}
	\begin{block}{}
		Тогда, математически, взаимодействие рассчитывается как усредненная разница при высоком и низком значении признака:
		\begin{equation}
			interaction = \dfrac{ (9 - 4) - (5 - 3)}{2} = 1.5
		\end{equation}

		Это значение нормируется на единицу изменчивости фактора (уже классический прием).
		Проверим симметричность влияния, зафиксировав другой фактор:
		\begin{equation}
			interaction = \dfrac{ (9 - 4) - (5 - 3)}{2} = 1.5
		\end{equation}

	\end{block}

	\note<+>{%
		Для начала рассчитаем влияние каждого фактора на эксперимент в отдельности (без учета взаимного влияния, аналогично предыдущему примеру).
		Затем, учтем фактор взаимного влияния, рассчитав изменения при одном фиксированном факторе (тип подсластителя).
		Тогда, математически, взаимодействие рассчитывается как усредненная разница при высоком и низком значении признака:
		$$
			interaction = \dfrac{ (9 - 4) - (5 - 3)}{2} = 1.5
		$$
		Это значение нормируется на единицу изменчивости фактора (уже классический прием).
		Проверим симметричность влияния, зафиксировав другой фактор:
		$$
			interaction = \dfrac{ (9 - 4) - (5 - 3)}{2} = 1.5
		$$
	}
\end{frame}
\begin{frame}[t]{Взаимодействие факторов}
	\begin{block}{}
		Таким образом влияние действительно симметрично и равнозначно. В итоге, наша модель запишется в виде:

		\begin{gather}
			\hat{y} = \dfrac{3 + 5 + 4 + 9}{4} + \dfrac{ (5 - 3) +
				(9 - 4)}{2} \cdot \dfrac{1}{2} \cdot x_A + \notag\\
			+\dfrac{(4 - 3) + (9 - 5)}{2} \cdot \dfrac{1}{2} \cdot x_B +
			\dfrac{1.5}{2} \cdot x_A x_B
		\end{gather}
		Или:
		\begin{equation}
			\hat{y} = 5.25 + 1.75x_A + 1.25x_B + 0.75x_Ax_B
		\end{equation}
	\end{block}

	\note<+>{%
		Таким образом влияние действительно симметрично и равнозначно. В итоге, наша модель запишется в виде:
		$$
			\hat{y} = \dfrac{3 + 5 + 4 + 9}{4} + \dfrac{ (5 - 3) +
				(9 - 4)}{2} \cdot \dfrac{1}{2} \cdot x_A +
			\dfrac{(4 - 3) + (9 - 5)}{2} \cdot \dfrac{1}{2} \cdot x_B +
			\dfrac{1.5}{2} \cdot x_A x_B
		$$
		$$
			\hat{y} = 5.25 + 1.75x_A + 1.25x_B + 0.75x_Ax_B
		$$

	}
\end{frame}


\begin{frame}[t]{Трехфакторный эксперимент}

	\begin{block}{}
		\begin{itemize}
			\item С (chemical), химическое соединение (два соединения P и Q).
			\item  T (temperature), температура очистки воды ( 72 F , 100 F ).
			\item  S (stirring speed), это скорость перемешивания (200 или 400 оборотов в минуту).
		\end{itemize}

		Количество экспериментов:
		\begin{equation}
			f^v = 3^2 = 8
		\end{equation}
		где f - число факторов, а v - число значений, принимаемых фактором.
	\end{block}
	\note<+>{%
		Новый пример взят из учебника Бокса, Хантера и Хантера, которая называется "Статистика для экспериментаторов".
		В этом эксперименте проводится поиск оптимальной комбинации параметров для уменьшения количества загрязнителя в сточных водах очистных сооружений.
		Рассматривается три фактора с 2 уровнями.

	}
	\note<+>{%
		\begin{itemize}
			\item С (chemical), химическое соединение (два соединения P и Q).
			\item  T (temperature), температура очистки воды ( 72 F , 100 F ).
			\item  S (stirring speed), это скорость перемешивания (200 или 400 оборотов в минуту).
		\end{itemize}

		Тогда количество необходимых экспериментов составит:
		$$
			f^v = 3^2 = 8
		$$
		где f - число факторов, а v - число значений, принимаемых фактором.

		Результатом эксперимента будет количество загрязняющих веществ, измеренное в фунтах. Используя стандартный порядок проведения эксперимента, составим таблицу эксперимента (табл. 4).

	}
\end{frame}

\begin{frame}[t]{Трехфакторный эксперимент}
	\begin{block}{}
		\center{\includegraphics[width=1\linewidth]{./images/statT2.png}}
	\end{block}
	\note<+>{%
		Одно из преимуществ такой таблицы заключается в том, что мы можем быстро
		получить общее представление о влиянии фактора на результат.
		Например, оцените как изменяется количество загрязняющих веществ, когда мы меняем фактор химического соединения C?
		Уровень фактора меняется с низкого на высокий и мы видим ту же самую картину с количеством загрязняющих веществ.
		Посмотрите на эффект фактора S.
		Первые четыре эксперимента в среднем показали очень высокий уровень загрязнения, а последние четыре эксперимента — низкий уровень загрязнения. Просто глядя на таблицу, мы можем сказать, что факторы C и S скорее всего важны для понимания полученных результатов.
		Общий вывод. Согласно полученным результатам, нам нужно взять химикат Q,
		проводить очистку при низкой температуре и на высокой скорости перемешивания (400 об/мин).

	}

\end{frame}

\begin{frame}[t]{Кубическая диаграмма}
	\begin{block}{}
		\center{\includegraphics[ width=0.9\linewidth]{./images/statG6.png}}
	\end{block}
	\note<+>{%
		Начнем с первого фактора C (выбора между химическими соединениями P и Q, где Q — высокий уровень фактора).
		Из кубической диаграммы мы можем получить четыре оценки эффекта С (вдоль каждого из четырех горизонтальных ребер).
		При высокой температуре и высокой скорости перемешивания (т.е. высокий уровень T и высокий уровень S) эффект от этого фактора равен 4-5 фунтов загрязнения.
		При высокой температуре и низкой скорости: 33-6.
		При низкой температуре и высокой скорости (т.е. Т - и S +), эффект равен: 3-4.
		И, наконец, при низкой температуре и низкой скорости: 30 и 5.
		Мы можем проанализировать полученную информацию с точки зрения каждого фактора и их возможного взаимодействия.

	}
	\note<+>{%
		В ходе опытов, химическое соединение показало четыре результата. Среднее для этих четырех чисел равно 50/4 = 12.5.
		Но что на самом деле означает полученное число 12.5?
		Как бы вы объяснили это значение своему менеджеру, который ничего не смыслит в статистике и экспериментах?

		Значение в 12.5 говорит о том, что в среднем мы ожидаем увидеть увеличение количества загрязняющих веществ на 12.5 фунтов на тонну при переходе от химического соединения P к Q (хотя для модели используется коэффициент 6.25 - половина).
		Таким образом для категориальных признаков в модели мы пишем половину от эффекта (учитываем нормировку).
		Еще одна вещь, на которую следует обратить внимание, это расхождение эффектов химиката при высоком и низком уровне перемешивания (S).
		Обратите внимание на огромную разницу, которая говорит о том, что существует явное взаимодействие между фактором C и фактором S.

	}
	\note<+>{%
		Прежде чем мы перейдем к взаимодействиям, давайте рассмотрим температуру (T).
		Заметное влияние температуры на отклик системы в таблице отсутствует.
		Это же подтверждает рассчитанный коэффициент в модели = 1.5 единицы (или 0.75 при нормировании эффекта).
		Это действительно слабый эффект.

		Наконец, рассмотрим эффект скорости перемешивания (S).
		Среднее для эффекта равно -14.5 (или -7.25 при нормировке).
		Другими словами, мы ожидаем среднее снижение количества загрязняющих веществ на 14.5 фунтов при переходе от низкой скорости перемешивания к высокой.

	}

\end{frame}

\begin{frame}[t]{Трехфакторный эксперимент}
	\begin{block}{}
		\begin{equation}
			\hat{y} = 11.25 + 6.25xC + 0.75xT - 7.25xS \dots
		\end{equation}

		\center{\includegraphics[ width=0.8\linewidth]{./images/statG6.png}}
	\end{block}

	\note<+>{%
		$$
			\hat{y} = 11.25 + 6.25xC + 0.75xT - 7.25xS \dots
		$$
		На этом этапе вам всегда надо делать паузу, чтобы убедиться, что полученные результаты имеют смысл.
		По горизонтальной оси мы видим, что переход от химиката P к Q увеличивает загрязнение (рис. 6).
		Поэтому значение 6.25 выглядит адекватно.
		Небольшое значение 0,75 для температуры также выглядит логично, потому что она действительно имеет очень слабый эффект.
		И, наконец, увеличение скорости перемешивания приводит к наиболее существенному снижению загрязнения: на 7.25 единиц.
		Примечание. Всегда проверяйте полученные коэффициенты модели на
		разумность!

	}
	\note<+>{%
		Как только мы закончили с интерпретацией факторов по отдельности, можно перейти к взаимодействиям.
		Ранее мы отметили, что эффект химиката сильно меняется при низкой скорости перемешивания.
		Однако на задней грани куба (при высоких скоростях перемешивания) эффект от выбора химиката практически равен нулю.
		Очевидно, что скорость перемешивания изменяет эффект от химического соединения.
		Таким образом мы наблюдаем взаимодействие между 2 факторами S и C.
		Для численной оценки воспользуемся уже знакомым нам приемом - добавим новый член в уравнение.

	}
	\note<+>{%
		У нас есть две возможности его рассчитать, фиксируя разные уровни переменной:

		1. при высокой температуре;
		2. вторая — при низкой температуре.

		Нет гарантии, что эффект будет симметричен, поэтому произведем оба расчета, а затем возьмем среднее (даже если эффект будет симметричен мы ничего не потеряем, а в противном случае - учтем оба влияния).
		А потом, как и всегда, нормируем на количество уровней признака (запишем половину).


	}
	\note<+>{%
		Пока что, мы учли только взаимодействие между факторами С-S.
		По остальным двухфакторным взаимодействиям не наблюдается видимого значимого влияния (одна из возможных причин - температура слабо влияет на модель).
		На самом деле, есть еще и трехфакторное взаимодействие C-T-S.
		Но пытаться все это учесть в ручную весьма утомительно и велик шанс наделать при этом ошибок.
		Далее мы будем использовать для этого компьютер. Поэтому пока остановимся на полученных результатов и проанализируем их.

	}
	\note<+>{%
		Общий анализ результатов.
		Основное заключение - при низких скоростях перемешивания химикат Q не эффективен, но при высоких оба химических соединения одинаково эффективны.
		Начиная с этого момента эксперименты становятся действительно мощным инструментом.
		Мы увидели, что самый низкий уровень загрязнений был при использовании химиката Q с высокой S и низкой T (найдите это значение на кубической диаграмме).
		Но что если, согласно требованиям правительства, загрязнение должно быть меньше 10?
		И при этом, допустим, химикат Q стоит вдвое дороже, чем P...
		На самом деле мы сейчас мысленно оценили дополнительный результат — прибыль.
		Не забывайте, что прибыль (или расходы) часто играют важную роль во всех системах.
		Поэтому вы всегда должны иметь в виду экономическую составляющую каждого угла куба.

	}
	\note<+>{%
	При этом мы убедились в малом эффекте температуры.
	И вот в чем вопрос: значит ли это, что рассматривать температуру в качестве фактора бессмысленно?
	И ответ - нет.
	Важно понимать, что даже незначительные эффекты представляют для нас важную информацию для изучения системы.
	Так, в нашем примере мы видим, что в диапазоне температур [70; 100]F температура оказывает незначительное влияние на количество загрязняющих веществ.
	И это важно, потому что на основании этой информации инженер или оператор может подобрать наиболее экономически выгодные условия работы.
	И, опять-таки, все сводится к прибыли.
	Вполне вероятно, что работа при более низкой температуре позволит сэкономить энергию.
	А поскольку температура оказывает лишь незначительное влияние на систему в целом, мы не окажем существенного влияния на уровень загрязнения если решим работать при низкой температуре.
	И это отличный результат.

	}
\end{frame}

{
\logo{}
\setbeamertemplate{background canvas}{\includegraphics[width=\paperwidth,height=\paperheight]{itmo/vitmo_snakes.jpg}}
\begin{frame}[standout]
	%Спасибо а внимание! 
	% \itmobackgroundsnakes{
	\vfill

	% \center\LARGE \contour{gray}{\color{white}{\bfseriesиСПАСИБО ЗА ВНИМАНИЕ!}}
	\center\LARGE {\color{white}{\bfseries СПАСИБО ЗА ВНИМАНИЕ!}}
	\vfill
	\includegraphics[width=0.4\textwidth]{itmo/slogan.pdf}
	% }
\end{frame}
}

\end{document}
